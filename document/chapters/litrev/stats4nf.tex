
Although the body of literature in the area of proposed research is not
expansive, there have been a number of relevant studies. These, in some way,
are related to the the prediction of forensics-applicable categories or
quantities of nuclear materials using statistical methods.  With regards to
broader forensics capabilities, materials from different steps of the nuclear
fuel cycle are being studied.  Even though each material has its own forensics
signatures, the process of applying statistical methods to the analysis of
material provenance is similar for each. 

For example, on the front end of the fuel cycle, an entity may have obtained
\gls{UOC} if they have enrichment capabilities.  One study performed
statistical analyses on \gls{UOC} from 21 sources (throughout seven countries)
using 30 concentration measurements of various elements, isotopes, and
compounds, e.g., sodium, magnesium, thorium, uranium-234, or halide compounds
\cite{robel_2009}.  The goal of classifying the source and the country was
reached 60\% and 85\% of the time, respectively, with the unique method
developed in the paper (an iterative partial least squares discrimination
analysis approach), which outperformed decision trees and \textit{k}-nearest
neighbors.
%Note: there was a ton of skew in their data and the correction for that was
%weak sauce

On the back end, an organization might have interest in \gls{SNF} if they have
reprocessing capabilities.  Or, perhaps already separated plutonium from
\gls{SNF} has been intercepted and needs to be traced. Another study addresses
this by performing factor analysis on theoretical separated plutonium from
various sources of \gls{ORIGEN}-simulated \gls{SNF} based on their composition
at the end of irradiation \cite{nicolaou_pu}.  Since in this study all
materials are the same age, five plutonium isotopes ($A = 238-242$) correctly
predicted a test sample. However, taking different times since irradiation and
reprocessing into account requires more isotopic measurements. 

\subsection{Factor Analysis Work}

In addition to the immediately aforementioned work, there is a suite of work on
performing classification using factor analysis or isotopic ratios in
combination with visual distinction.  Although factor analysis explicitly
requires the input of domain knowledge, it is a valuable first step towards
understanding how statistical methods can provide insightful models to classify
materials.  This series of work seeks to accomplish classification of a reactor
history (reactor type, enrichment, burnup) using the distinguishability between
similar sets of samples, as pictured in Figure \ref{fig:nico}.

\begin{figure}[!htb]
  \centering
  \includegraphics[width=0.5\linewidth]{./chapters/litrev/nicolaou.png}
  \caption[Example of factor analysis for \acrshort{SNF} discrimination]
          {Example from Reference \cite{nicolaou_pu} on using factor analysis
           to show similarities between classes of \acrshort{SNF} for visual 
           identification.}
  \label{fig:nico}
\end{figure}

The chronologically earliest study in this grouping of publications
\cite{nicolaou_2006} uses \gls{ORIGEN}-simulated uranium and plutonium
(${}^{234}\text{U}$, ${}^{235}\text{U}$, ${}^{236}\text{U}$,
${}^{238}\text{U}$, ${}^{238}\text{Pu}$, ${}^{239}\text{Pu}$,
${}^{240}\text{Pu}$, ${}^{241}\text{Pu}$ and ${}^{242}\text{Pu}$) of various
\gls{SNF} designations.  These simulated measurements are subject to factor
analysis and the results usually plotted.  The unknown samples in this work are
in visual alignment with the factor-analysis-determined groupings.  Reference
\cite{nicolaou_2014} extends that study (same statistical method, same list of
isotopes) to real measured samples from the \gls{SFCOMPO} database
\cite{sfcompo, valid_sfco}.  The goal here was to determine whether the factor
analysis approach could distinguish the chosen \gls{SFCOMPO} entries well, and
the results confirm the goal was achieved.  Reference \cite{nicolaou_2015} uses
three plutonium ratios (${}^{242}\text{Pu}$/${}^{240}\text{Pu}$,
${}^{238}\text{Pu}$/$\text{Pu}_{\text{Total}}$, and
${}^{239}\text{Pu}$/${}^{240}\text{Pu}$) to accomplish the same goal with
simulated \gls{SNF} without factor analysis, and the results are similar. 

This work mostly relies on actinides for identifying \gls{SNF}
\cite{nicolaou_2006, nicolaou_pu, nicolaou_2014, nicolaou_2015}, but the use of
fission products is also promising \cite{nicolaou_2009}.  This particular
publication was interesting because the author chose to use one set of fission
products to represent a typical mass spectrometry assay (${}^{133}\text{Cs}$,
${}^{140}\text{Ce}$, ${}^{150}\text{Sm}$, ${}^{152}\text{Sm}$,
${}^{144}\text{Nd}$, ${}^{145}\text{Nd}$, ${}^{146}\text{Nd}$,
${}^{148}\text{Nd}$, ${}^{150}\text{Nd}$), and another set to represent what
could be determined from a gamma detector (${}^{95}\text{Zr}$,
${}^{95}\text{Nb}$, ${}^{106}\text{Ru}$, ${}^{134}\text{Cs}$,
${}^{137}\text{Cs}$, ${}^{144}\text{Ce}$).  This approach is successful when
the simulation uncertainty is below 3\% \cite{nicolaou_2009}.

\subsection{Other Classification Work}

There are other papers on statistical methods that focus on the classification
of the reactor type for unknown samples.  One study simulated and tracked 34
nuclides of a set of typical commercial nuclear power reactors and their
operation parameters, but first used statistical dimensionality reduction (via
Laplacian eigenmaps) before subjecting the training data to reactor type
classification, comparing linear discriminant analysis, quadratic discriminant
analysis, random forests, and Parzen window classifiers \cite{jones_snf_2014,
jones_viz_2014}.  Condensing the 34 features into three has not only
computational and potentially discriminatory benefits, there are visualization
benefits as shown in Figure \ref{fig:jones}.  This plot and this work also
highlights and addresses a known problem: reliable discrimination between
\gls{SNF} from \glspl{PWR} and \glspl{BWR}. 

\begin{figure}[!htb]
  \centering
  \includegraphics[width=0.7\linewidth]{./chapters/litrev/jones.png}
  \caption[Example of dimensionality reduction for visualization]
          {Results of leveraging a pre-processing step of dimensionality 
           reduction for visualization purposes, from Reference 
           \cite{jones_snf_2014}.}
  \label{fig:jones}
\end{figure}

Another paper compared principal components analysis and partial least squares
discriminant analysis to classify reactor type \cite{pu_discrimination}. As
with the above, a set of \gls{SNF} from typical commercial power reactors from
around the world were simulated and used as test samples for this attribution
step, but unlike the above they chose to focus on uranium and plutonium
isotopes to address the challenge of identifying chemically separated uranium
or plutonium. This work uses a qualitative visual distinguishability discussion
to choose the better method of partial least squares discriminant analysis.

\subsection{Regression Work}

Switching the focus to regression, one work combines the use of statistical
methods with an investigation of those methods when faced with information
reduction via random nuclide measurement errors in the training data set
\cite{dayman_feasibility_2013}.  Additionally, feature reduction was
investigated by using various nuclide compositions: the top 200 nuclides by
concentration in each vector, fission products only, and a principal components
analysis-derived shortened nuclide list.  

\begin{figure}[!htb]
  \centering
  \includegraphics[width=\linewidth]{./chapters/litrev/dayman.png}
  \caption[Burnup performance with respect to training set error]
          {Plot from Reference \cite{dayman_feasibility_2013} that shows the 
           degradation of burnup prediction performance with respect to 
           increasing training set error for three algorithm implementations 
           and three feature sets.}
  \label{fig:dayman}
\end{figure}

Using these various feature sets, three methods were also compared. First and
second, the nearest neighbor algorithm with two different distance metrics was
used: Manhattan distance ($L_1$ norm, or sum of absolute differences of
Cartesian coordinates) and Euclidean distance ($L_2$ norm, or square root of
the sum of squared differences of Cartesian coordinates).  The nearest neighbor
approaches classified reactor type and predicted burnup.  Third, ridge
regression with an $L_2$ norm for regularization was only applied to burnup
prediction.  In both classification and regression cases, using the fission
products nuclide list with both nearest neighbor methods performed the best.
All other nuclide lists quickly devolved to random guesses with an increase in
nuclide error in the case of reactor prediction, and more than 100\% error in
the case of burnup prediction. Figure \ref{fig:dayman} shows the behavior of
burnup prediction at low training set error (under 1\%) for the various methods
with feature set combinations.

\subsection{Maximum Log-Likelihood Calculations}

Another set of publications focuses on a novel methodology for attributing
separated plutonium with different reactor histories.  Although many commonly
used statistical methods have been previously discussed in this section, this
approach is unique: a \gls{MLL} calculation approach for determining the
similarity of a test sample to a database of simulated samples
\cite{mll_method}.  

\begin{figure}[!htb]
  \centering
  \includegraphics[width=0.6\linewidth]{./chapters/litrev/tamu.png}
  \caption[Example of likelihood maximum predicting burnup and time since 
           irradiation]
          {Plot from Reference \cite{mll_method} that shows a sample being 
           assigned burnup and time since irradiation values based on a 
           likelihood maximum.}
  \label{fig:tamu}
\end{figure}

\begin{figure}[!htb]
  \centering
  \includegraphics[width=\linewidth]{./chapters/litrev/tamu2.png}
  \caption[Sensitivity of likelihood to training set uncertainty]
          {Results from a sensitivity study on the \gls{MLL} method where 
           uncertainty of the training set was increased. The levels shown 
           here are (a) 7\%, (b) 14\%, (c) 21\%, and (d) 28\% 
           \cite{mll_sensitivity}.}
  \label{fig:tamu2}
\end{figure}

Since this approach was developed to attribute weapons-grade plutonium, the
training database is simulated with low burnup values, and there are 5000
one-day time steps.  The reactor type, burnup, and time since irradiation
comprise the labels, and the features are a set of 10 carefully selected
isotope ratios. 

Figure \ref{fig:tamu} shows one of the results, where a simulated sample of
$4.39\:GWd/MTU$ burnup and $3652\:days$ time since irradiation is tested
against the training database via likelihood calculations of the feature set of
10 isotope ratios; the maximum is visibly close to the ground truths.  This
method was later validated with experimental samples, in Reference
\cite{mll_validate}. 

After experimental validation, sensitivity studies were conducted in Reference
\cite{mll_sensitivity}.  Figure \ref{fig:tamu2} shows a set of results from the
increasing of uncertainty for the same sample being predicted in Figure
\ref{fig:tamu}.  It is interesting that the likelihood decrease happens faster
on the time since irradiation axis than on the burnup axis. The main result
from the sensitivity studies is that while a different sample was robust to the
increased uncertainty and was predicted at the 99\% confidence level even at
the highest uncertainty (28\%), the sample shown here was predicted at only a
68\% confidence level at the highest uncertainty.

\subsection{Summary}

The factor analysis works \cite{nicolaou_2006, nicolaou_pu, nicolaou_2009,
nicolaou_2014, nicolaou_2015} and other visualization- and classification-based
works \cite{pu_discrimination, jones_snf_2014, jones_viz_2014} helped provide a
foundation by which to understand the scope of problem of attributing
\gls{SNF}. Additionally, some of these use the \gls{SFCOMPO} database in their
study design \cite{nicolaou_2014, jones_viz_2014}. While Reference
\cite{nicolaou_2014} uses the factor analysis method directly on some subset of
\gls{SFCOMPO} database entries, Reference \cite{jones_viz_2014} uses nine
samples from the database as test cases for both reactor type classification
and burnup/${}^{235}\text{U}$ enrichment.  This work instead tests all
available \gls{SFCOMPO} entries after some filtering steps are carried out to
remove cases that are not approximated by the training set.  Reference
\cite{nicolaou_2009} chooses a set of nuclides based on a comparison of mass
spectrometry to gamma spectroscopy, which also happens in this work. However,
there is no indication that there is a gamma detector-based treatment of the
chosen fission products. It is presumed they are using mass- or activity-based
values to represent the features in their training set. 

Reference \cite{robel_2009} performs prediction of \gls{UOC} provenance,
comparing a newly developed method against two simple algorithms; this informed
the method of choosing of algorithms in this work. It is important to first
create a benchmark a simple approach before scaling up to more complex
treatments of the data. The approach in this reference used \textit{k}-nearest
neighbors, decision trees, and a unique iterative method.  It is actually a
coincidence that the first two methods match the ones chosen for this work,
because other methods were tested before deciding to use both
\textit{k}-nearest neighbors and decision trees. Still, this work chooses a
different third method to implement and shifts the focus to \gls{SNF}. 

This third method is the \gls{MLL} calculations developed in References
\cite{mll_method, mll_validate, mll_sensitivity}.  The mathematical framework
for \gls{MLL} calculations is in Section \ref{sec:algs}.  There are a few
differences in implementation. This work focuses on \gls{SNF} instead of a
material source presumed to be separated plutonium.  Additionally, the time
steps in the references are much smaller than what exists in the training set
in this work, and these publications focus on a set of 10 isotope ratios as the
feature set.  Also, instead of studying two well-characterized samples (that
were at first simulated but then irradiated and real measurements were taken),
this work instead is focused on the aggregate statistics of many predictions.
Despite these differences, this approach is still applicable to and an asset
for this work. 

A paper that greatly influenced the development of this study
\cite{dayman_feasibility_2013} seeded the decision to evaluate the effect of
information reduction on the predictive capabilities of the statistical methods
used.  While the first experiment in Chapter \ref{ch:exp1} uses a similar
application of random error, the experiment in Chapter \ref{ch:exp2} instead
uses detectors with decreasing energy resolution to accomplish information
reduction.

In summary, there is limited treatment of statistical methods predicting any
nuclear material's attributes when facing information reduction in the
literature.  There is also limited use of the \gls{SFCOMPO} database as a
testing set.  The first experiment in Chapter \ref{ch:exp1} addresses these
thin areas of previous work.  There is no work to the author's knowledge using
a gamma detector-based treatment for the measurement of nuclides for use in
statistical approaches, and the second experiment in Chapter \ref{ch:exp2}
concentrates on this.
