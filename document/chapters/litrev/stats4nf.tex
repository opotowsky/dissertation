Although the body of literature on the area of proposed research is not
expansive, there have been a number of relevant studies on the prediction of
forensically relevant categories or quantities of nuclear materials using
statistical methods. 

\subsection{Special Nuclear Materials Studied}

With regards to broader forensics capabilities, materials from different steps
of the nuclear fuel cycle are being studied.  Even though each material has
its own forensics signatures, the process of applying statistical methods to
the analysis of material provenance is similar for each. 

For example, on the front end of the fuel cycle, an entity may have obtained
\gls{UOC} if they have enrichment capabilities.  One study performed
statistical analyses on \gls{UOC} from 21 sources (throughout seven countries)
using 30 concentration measurements of various elements, isotopes, and
compounds, e.g., sodium, magnesium, thorium, uranium-234, or halide compounds
\cite{robel_2009}.  The goal of classifying the source and the country was
reached 60\% and 85\% of the time, respectively.  
%Note: there was a ton of skew in their data and the correction for that was
%weak sauce

On the back end, an organization might have interest in \gls{SNF} if they have
reprocessing capabilities.  Or, perhaps already separated plutonium from
\gls{SNF} has been intercepted and needs to be traced. Another study addresses
this by performing factor analysis on theoretical separated plutonium from
various sources of \gls{ORIGEN}-simulated \gls{SNF} based on their composition
at the end of irradiation \cite{nicolaou_pu}.  Since in this study all
materials are the same age, five plutonium isotopes ($A = 238--242$) correctly
predicted a test sample. However, taking different times since irradiation and
reprocessing into account requires more isotopic measurements. 

\subsection{Statistical Methods Employed}

There are statistical methods studies that focus on the classification of the
reactor type for unknown samples \cite{robel_2009, nicolaou_pu, jones_snf_2014,
nicolaou_2009}.  However, this work is focused on burnup prediction.  Although
the results for both regression and classification are based on a number of
features that are usually isotopic in nature, it is not clear if the regression
counterparts of these algorithms will perform similarly for this task. 

Promising regression work using factor analysis has been published
\cite{nicolaou_2006, nicolaou_2014}.  Although factor analysis explicitly
requires the input of domain knowledge, it is a valuable first step towards
understanding how statistical methods can provide insightful models that
predict fuel enrichment and burnup. In the following two cases, the features
included for the analysis are only the uranium and plutonium isotopes remaining
in the \gls{SNF}.  Ref. \cite{nicolaou_2006} covers predicting enrichment and
burnup from a range of simulated \gls{SNF} recipes and comparing an `unknown'
sample to the results of the factor analysis. Ref.  \cite{nicolaou_2014}
extends that study to real measured samples from the \gls{SFCOMPO} database
\cite{sfcompo}. This work also highlights and addresses a known problem:
reliable discrimination between \gls{SNF} from \gls{PWR}s and \gls{BWR}s. 

The most closely related work to this study involves not only statistical
methods but an investigation of those methods when faced with information
reduction via random nuclide measurement errors in the training data set
\cite{dayman_feasibility_2013}.  Additionally, feature reduction was
investigated by using various nuclide compositions: the top 200 nuclides by
concentration in each vector, fission products only, and a \gls{PCA}-derived
shortened nuclide list.  Three methods were compared. First and second, the
nearest neighbor algorithm using both $L_1$ (sum of absolute differences of
Cartesian coordinates) and $L_2$ norms for measuring distance between test data
points classified reactor type and predicted burnup.  Third, ridge regression
with an $L_2$ norm for regularization was only applied to burnup prediction. In
both classification and regression cases, using the fission products nuclide
list with both nearest neighbor methods performed the best. All other nuclide
lists quickly devolved to random guesses with an increase in nuclide error in
the case of reactor prediction, and more than 100\% error in the case of burnup
prediction.

For reliable prediction, it seems to be promising to use actinides
\cite{nicolaou_2006, nicolaou_2014} and/or fission products
\cite{dayman_feasibility_2013} for domain knowledge-based feature reduction.
However, this work intends to still investigate statistical methods for
dimensionality reduction, e.g., \gls{PCA}. This could be beneficial in
prediction of burnup or other reactor parameters, or could be useful in other
ways, such as visualization or discovering new correlations among \gls{SNF}
properties as new reactor technologies are deployed.

