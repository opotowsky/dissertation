\setlength\abovedisplayskip{2.5pt}

For relevant nuclear forensics predictions, both classification and regression
algorithms must be used.  For example, one may want to predict the reactor type
label given some measurement-based features of \gls{SNF} of an unknown source.
This would require a classification algorithm. Or perhaps the input fuel
composition is relevant to an investigation on weapons intent, so a regression
algorithm would be used. 

There are three algorithms presented in this section: \textit{k}-nearest
neighbors, decision trees, and \gls{MLL} calculations. They were chosen based
on their simplicity; this work has yet to be benchmarked using simple
algorithms so a more complex treatment of the training sets in this work would
be premature. Additionally, in part because of their simplicity, they are all
"white box" methods.  This is unique in the \gls{ML} universe, since most
algorithms create a black box model that is unable to be analyzed by a human.
The  decision trees method provides an output model that can be used to discern
behavior and understand predictions, and \textit{k}-nearest neighbors and
\gls{MLL} calculations do not create a model at all. Individual predictions can
still be analyzed, however, since the procedures are so simple. 

\subsubsection{Nearest Neighbor Methods}

Nearest neighbors classification and regression are unique algorithms in
that thay are instance-based; they do not actually generalize, but instead
track the observations in the training set.  The main metric for this algorithm
is distance (or dissimilarity) between the test sample and the closest training
sample(s) in the vicinity.  During prediction, the algorithm will calculate a
value based on the instance that is closest to the current test sample. Thus,
there is not any learning, but instead a direct comparison between an unknown
sample and the space that the training set populates. The predictions from
nearest neighbors can be quite accurate, but are highly unstable to
peturbations \cite{elements_stats}.

Prediction with this method will be explained using \textit{k}-nearest
neighbors regression.  First, the distances between the test sample and each of
the training set instances are calculated.  Most commonly the Euclidian
distance is used, but this walkthrough uses the Manhattan distance:
\begin{equation}
  d_{i} = \sum_{j=1}^{N_{feats}} |x_{j,train} - x_{j,test}|
  \label{eq:l1}
\end{equation}
where $i$ is each training set instance, and $j$ refers to each feature in the
training set.  The lowest \textit{k} $d_{i}$ are chosen, and the value, $Y$ is
predicted using:
\begin{equation}
  Y(\boldsymbol{x}) = \frac{1}{k} \sum_{i=1}^{k} w_i \cdot y_i
  \label{eq:knn}
\end{equation}
where $w_{i}$ is either uniform and takes on a value of $1$ or is
distance-based and takes on a value of $1/d_{i}$ and $\boldsymbol{x}$ is the
full set of features.  This averages the closest \textit{k} neighbors for an
estimate of the unknown sample, as shown in Equation \ref{eq:knn}.  For
classification, the mode of the \textit{k} nearest neighbors is taken.  

\begin{figure}[!htb]
  \centering
  \includegraphics[width=0.8\linewidth]{./chapters/litrev/nn-fig.png}
  \caption{Schematic of \textit{k}-nearest neighbors regression, showing how 
           changing \textit{k} alters the predicted label value $y$.}
  \label{fig:nn}
\end{figure}

Figure \ref{fig:nn} provides a pictoral explanation of Equation \ref{eq:knn}
for a prediction where there is one feature. In this figure, there is a test
sample with a feature, valued at $x_i$, indicated with the grey dotted line.
The three circles represent the neighborhood given by the value of \textit{k},
and the darker dots on the line represent the reported prediction $y$ for each
choice of \textit{k}.  In this illustration, $k=1$ or $k=2$ provide a more
accurate prediction according to a visual inspection of the trend, but higher
values of $k$ can be useful, and will be discussed in Section
\ref{sec:complexity}.

\subsubsection{Decision Trees}

The predictions from decision trees, similar to \textit{k}-nearest neighbors,
are highly unstable to peturbations. 

\cite{elements_stats}

\subsubsection{Maximum Log-Likelihood Calculations}

The \gls{MLL} calulations approach applied here is based on a method developed
to do similar work \cite{mll_method, mll_validate, mll_sensitivity}.  That work
involved matching nuclear material samples based on some select measurements to
entries in a database of containing those measurements.\todo{keep vague or no?}
Each database entry also has a similar list of labels to the labels being
predicted in this work: reactor type, burnup, and time since irradiation.

Interestingly, the \gls{MLL} calculations method works like \textit{k}-nearest
neighbors, where there is no model but a prediction according to the closest
match database entry.  There is one detail that differs, however. Whereas
\textit{k}-nearest neighbors minimizes distance/dissimilarity, this approach
instead maximizes similarity via a likelihood function. An "unknown" test
sample is compared against the training set using the likelihood calculation
between that sample and the training set entries.  The higher the likelihood,
the higher the probability that the database entry represents the sample. The
likelihood is in Equation \ref{eq:like}, whereas the log-likelihood is used
more often in practice, shown in Equation \ref{eq:loglike}.
\begin{equation}
  L(M|x_{test}) = \prod_i \frac{1}{\sigma_{i,train} \sqrt{2\pi}} \exp{\frac{-(x_{i,test} - x_{i,train})^2}{2 \sigma_{i,train}^2}}
  \label{eq:like}
\end{equation}
\begin{equation}
  ln(L(M|x_{test})) = \sum_i ln(\frac{1}{\sigma_{i,train} \sqrt{2\pi}}) - \frac{(x_{i,test} - x_{i,train})^2}{2 \sigma_{i,train}^2}
  \label{eq:loglike}
\end{equation}
The likelihood is a measure of the probability that a model $M$ produced the
measurements seen in the test sample, given by $L(M|x_{test})$.  In both
Equations \ref{eq:like} and \ref{eq:loglike}, $x$ refers to the set of
features, and $x_{i, test}$ and $x_{i,train}$ are the individual features for the
test sample and the training set entries, respectively. The uncertainty of the
measurement associated with each feature is represented by $\sigma_{i,train}$.

