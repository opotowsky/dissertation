
\begin{figure}[H]
  \centering
  \includegraphics[width=0.7\linewidth]{./chapters/exp2/methodology2_2.png}
  \caption[Second portion of the flowchart from Figure \ref{fig:method2}]
          {Second portion of the flowchart from Figure \ref{fig:method2} being 
           described in this section.}
\end{figure}

The overall goal of this project is to determine how much information to what
quality is needed to train a  \gls{ML} model that can provide \gls{SNF}
attribution by correctly predicting the reactor type, burnup, \gls{U235}
enrichment, and time since irradiation.  In this section, the information
quality is treated as the energy resolution of gamma spectra from several
detectors.  This is because field-deployable detectors are of interest.

This process is outlined here for the second experiment, in which a gamma
spectrum is computed for each sample in the database from the nuclide
activities in Section \ref{sec:training2}.  The \gls{GADRAS} code \cite{gadras}
developed at Sandia National Laboratories will provide computational gamma
spectra. There are four steps to this process: choosing radionuclides,
inputting the radionuclide activities into a \gls{DRF} to compute gamma
spectra, processing the gamma spectra into a smaller feature set, and applying
a statistical counting error to those features.

\noindent \textbf{Step 1: Choosing Radionuclides}

The first step of is covered in Section \ref{sec:training2}.  

\noindent \textbf{Step 2: Computational Gamma Detection}

The second step is obtaining a gamma spectrum for every \gls{SNF} entry in the
database; this is done using \gls{GADRAS}. The \gls{GADRAS} tool includes many
capabilities related to radiation detection and spectra analysis; its
predominant use is related to the simulation of the detection of gamma rays and
neutrons from user-defined sources and detector configurations.  A combined
first principles and empirical modeling approach based on interaction cross
sections and radiation scattering inform the \glspl{DRF} code to calculate
typical gamma detector spectrum features, e.g., photopeaks and the Compton
continuum.  Although new detectors can be created and calibrated within the
software, this work employs the use of pre-calibrated detectors.  Thus, given a
source, which is a list of radionuclides and their activities in this case,
\gls{GADRAS} applies a \gls{DRF} from a given detector configuration to the
gamma energy lines from these radionuclides and models the spectrum.
\cite{gadras} 

The nuclide activity data requires some processing to be used in this way.  The
activities that come from the \gls{ORIGEN} simulations are based on there being
$1\:MT$ of initial uranium-based fuel. Not only is this quantity an unlikely
amount to be smuggled, it would overwhelm a detector at the calibrated
source-detector distances in \gls{GADRAS}.  Therefore, the material (and
resulting nuclide activities) are scaled to be $1\:g$ of \gls{SNF}.

For the \gls{GADRAS} calculations, the primary input is a nuclide activity
vector as the source, and the output is an array of energy bins (measured in
$keV$) and the counts per energy bin as the spectrum.  The sources are provided
without any background; this is because any spectrum would undergo background
subtraction before further analysis. Additionally, the nuclides are pre-decayed
in \gls{ORIGEN} to correspond to various cooling times, but a source age must
be provided to \gls{GADRAS} as a non-zero value (otherwise, important decay
transitions are missed). Various source times ($10\:sec$, $30\:sec$, $1\:min$,
$5\:min$, $10\:min$, $20\:min$, $30\:min$, $1\:hr$, $2\:hr$) were tested for
two samples, and both a qualitative visual analysis of the major photopeaks
plus a study of the total counts leveling off was used to choose a nonzero age
for the material.  At a source age of $20\:minutes$, the expected peaks are
visible and the total counts had reached a value similar to the longer ages
tested.

The other input information is related to the detector configuration.
User-chosen variables are the source-detector distance $[cm]$, height of
source-detector setup from nearest surface $[cm]$, the live time $[s]$, and the
number of channels for the detector. The detector configuration file in
\gls{GADRAS} contains much more information, however.  In it, there are a
number of parameters from calibration results, detector geometry, information
about scattering, and shielding information (shielding is not considered in
this work).

\begin{table}[!htb]
  \centering
  \begin{tabular}{@{}lcllll@{}}
  \toprule
    \textbf{Detector} &
    \textbf{\begin{tabular}[c]{@{}c@{}}\% FWHM \\ @ 661 keV\end{tabular}} &
    \textbf{\begin{tabular}[c]{@{}l@{}}Distance \\ (cm)\end{tabular}} &
    \textbf{\begin{tabular}[c]{@{}l@{}}Height \\ (cm)\end{tabular}} &
    \textbf{\begin{tabular}[c]{@{}l@{}}Live Time\\ (s)\end{tabular}} &
    \textbf{\begin{tabular}[c]{@{}l@{}}Num \\ Channels\end{tabular}} \\ \midrule
    In-Lab HPGe           & 0.21 & 100.0 & 84.0  & 600  & 8192 \\
    Portable HPGe         & 0.29 & 100.0 & 100.0 & 600  & 8192 \\
    CZT                   & 1.20 & 100.0 & 100.0 & 600  & 1024 \\
    SrI\textsubscript{2}  & 2.94 & 100.0 & 100.0 & 600  & 1024 \\
    LaBr\textsubscript{3} & 3.63 & 213.0 & 84.5  & 2400 & 1024 \\
    NaI                   & 7.74 & 213.0 & 85.4  & 2400 & 1024 \\ \bottomrule
  \end{tabular}
  \caption[Details of detector setups]
          {Select details of 6 detector setups used to obtain gamma 
           spectra-based training databases.}
  \label{tbl:detsetups}
\end{table}

Training databases were created for the six detectors outlined in Table
\ref{tbl:detsetups}. They were chosen to compare the highest energy resolution
detector, a lab-based \gls{HPGe}, against the rest, in order of decreasing
energy resolution: portable \gls{HPGe}, \gls{CZT}, \gls{SrI2}, \gls{LaBr3}, and
\gls{NaI} detectors. This is displayed in the table by including the \gls{FWHM}
of the $661\:keV$ peak for ${}^{137}\text{Cs}$. \Gls{GADRAS} is used to create
a spectrum for every \gls{SNF} entry in the 32 nuclide activity training
database using these six detector configurations.  At this point, there are six
versions of the original database, one for each detector.  The database entries
are each a full gamma spectrum of a given \gls{SNF} sample for the detector
setup in that training set. 

\noindent \textbf{Step 3: Processing Gamma Spectra}

The third step covers the processing of the gamma spectrum generated for each
\gls{SNF} entry into a training set.  It is not computationally prudent to use
full gamma spectra for training and testing, since the spectra returned have
1024 or 8192 bins, and machine learning algorithms are not designed to handle
thousands of features.  Thus, these spectra are processed into fewer features; 
this is outlined as follows.

There is much ongoing work on the topic of attributing \gls{SNF} with gamma
detection using targeted or advanced measurement techniques \cite{snf_gamma,
compton_supp, bwr_high-res_gamma, pwr_bwr_gamma} and using innovative spectra
evaluation and radioisotope identification methods \cite{riid_09,
rapid_riid_18, sull_gen_07, sull_valid_15, sull_auto_17, sull_unc_17}.  Since
this is a topic of active research and the approaches are heavily
detector-dependent, a simple processing approach is developed here.  Since all
entries in a given training set are background-subtracted spectra using the
same detector setup and calibration for the same measurement duration, each
photopeak can be directly compared to other photopeaks in the training set.
This is accomplished by comparing them via an area under the curve: placing an
energy window on a peak, and summing the counts of the bins within that energy
window. 

There are two main design choices here: the width of the energy windows and the
number of energy windows to include. First, the energy window width is a value
that is fixed for each detector prior to processing.  The different energy
window widths are listed in Table \ref{tbl:enwindows}.  They are chosen
manually; the spectra were plotted and different values were tested and
visually analyzed to be sure the windows were encompassing the peaks. This was
preferable to some linear function based on the detector energy resolution
because, e.g., the \gls{LaBr3} and \gls{NaI} detectors have the same
human-chosen window but the energy resolutions are different (see Table
\ref{tbl:detsetups}). 

\begin{table}[!htb]
  \centering
  \begin{tabular}{@{}lcm{0.7in}m{0.7in}m{0.7in}@{}}
    \toprule
    \multirow{2}{*}{\textbf{Detector}} &
    \multirow{2}{*}{\textbf{\begin{tabular}[c]{@{}l@{}}Energy Window\\ Size {[keV]}\end{tabular}}} &
    \multicolumn{3}{c}{\textbf{\# of Energy Windows}} \\ \cmidrule(l){3-5}
                     &    & Auto & Short & Long \\
    \toprule
    In-Lab HPGe      & 2  & 206  & 42    & 151  \\
    Portable HPGe    & 3  & 120  & 42    & 151  \\
    CZT              & 8  & 30   & 42    & 151  \\
    $\text{SrI}_2$   & 10 & 17   & 42    & 151  \\
    $\text{LaBr}_3$  & 12 & 19   & 42    & 151  \\
    NaI              & 12 & 9    & 42    & 151  \\ 
    \bottomrule
  \end{tabular}
  \caption[Energy window sizes and list lengths for processing gamma spectra]
          {Energy window sizes and list lengths for 6 detector setups used to 
           process the gamma spectra-based training databases.}
  \label{tbl:enwindows}
\end{table}

The second design decision is the number of energy windows to include.  Table
\ref{tbl:enwindows} lists three energy window list length columns,
\textit{Auto}, \textit{Short}, and \textit{Long}. These correspond to different
processed training sets that have a different number of energy windows
included.  There are two approaches taken: a nuclear physics-based method that
generates an energy window list based on the gamma energies expected to be
detected (short and long), and an automatic peak search of a manually chosen
gamma spectrum in the full gamma spectra training database (auto). 

The first energy window list method provides the short and long lists in Table
\ref{tbl:enwindows}; the length of these lists are the same for all detectors
because they are based on the gamma energies most likely to be detected, which
is independent of the detector quality. To obtain these lists, the expected
number of decays of each gamma energy is calculated based on the activities of
the 32 tracked nuclides using the Python for Nuclear Engineering toolkit
\cite{pyne} from a reference sample in the training set.  This reference sample
emerged as the sample of choice because it contains the superset of gamma
energies of all tested samples, of which there were nine (three for each
reactor type).

After the expected number of decays for each gamma energy for all 32 nuclides
are calculated, an arbitrary minimum number of decays is selected to filter out
the gamma energies that are unlikely to produce counts high enough to be
detected. The first arbitrary minimum number of decays is set at $5 \times
10^8$ decays, and the long list of 151 gamma energies that remain above this
cutoff is created.  A list of length 151 is likely to contain many features
that do not contribute discrimination to the models, and thus they may add
noise to the training set, so a shorter list is needed.  So, a higher arbitrary
minimum of $5 \times 10^{10}$ decays is chosen; this threshold creates the
short list of 42 gamma energies. 

The short and long lists of gamma energies correspond to the nuclides in Table
\ref{tbl:enlistnucs}. The 12 nuclides listed come from the long list, and the
subset of seven bold nuclides come from the short list.  To determine a "full
knowledge" scenario for the detector-based training sets, two training sets are
also created with the 7- and 12-nuclide activity lists.  It should be noted
that several (three for each reactor type) training set entries were selected
based on sampling evenly throughout the training set parameters, with the
intention that there would be a set of gamma energies comprised from multiple
entries. However, one sample emerged as a superset of the others. This sample
is thus chosen for the second method, discussed next.

\begin{table}[!htb]
  \centering
  \begin{tabular}{@{}|l|l|l|@{}}
    \hline
    \allbold{${}^{241}\text{Am}$} & \allbold{${}^{243}\text{Am}$} & ${}^{243}\text{Cm}$           \\ \hline
    ${}^{244}\text{Cm}$           & ${}^{245}\text{Cm}$           & \allbold{${}^{134}\text{Cs}$} \\ \hline
    \allbold{${}^{137}\text{Cs}$} & ${}^{152}\text{Eu}$           & \allbold{${}^{154}\text{Eu}$} \\ \hline
    \allbold{${}^{85}\text{Kr}$}  & ${}^{238}\text{Pu}$           & \allbold{${}^{125}\text{Sb}$} \\ \hline
  \end{tabular}
  \caption[Nuclides that are represented by the short and long energy windows 
           lists]
          {Nuclides that are represented by the gamma energy lines in the 
           energy lists. The entire set of 12 nuclides belongs to the long 
           list, and the 7 bold nuclides belong to the short list.}
  \label{tbl:enlistnucs}
\end{table}

The column denoted as \textit{Auto} in Table \ref{tbl:enwindows} is obtained by
a physics-free approach. It is based on a peak search of a spectrum in the full
gamma spectra training database. The previously mentioned sample is selected
for all six detectors, and a peak searching algorithm implemented in python
using the SciPy toolkit \cite{scipy} is applied. Using the peak search on the
six different spectra for the same sample, the resulting number of energy
windows for each detector is in Table \ref{tbl:enwindows}.  

%this is idx 66796
\begin{figure}[!htb]
  \makebox[\textwidth][c]{\includegraphics[width=\linewidth]{./chapters/exp2/energy_window_example.png}}
  \caption[Portion of gamma spectrum with windows, which shows summation regions 
           for spectra processing]
          {Slice of an example gamma spectrum in one of the training databases
           showing the windows over the gamma energy peaks. This is the portable
           \acrshort{HPGe} with the auto energy window list and the reference 
           sample's spectrum.}
  \label{fig:enwindows}
\end{figure}

After the three energy window lists are created, the full gamma spectra
database of a given detector are processed into three training sets, one for
each list.  Next, as previously mentioned, the energy window width for each
detector is used to sum the binned counts for each energy window list entry.
this is visualized in Figure \ref{fig:enwindows}, where a portion of a portable
\gls{HPGe} spectrum is shown with the $\pm3\:keV$ windows from the auto energy
window list.  Three training sets are created for each detector, resulting in
18 detector-based processed gamma spectra training sets.

\noindent \textbf{Step 4: Apply Statistical Counting Error}

Lastly, the fourth step involves the inclusion of the counting error for the
summed energy windows. This is quite simple mathematically speaking, as
statistical counting error of $n$ counts is $\sqrt{n}$.  As in Section
\ref{sec:inforeduc1}, this error gets applied in the same way for the
scikit-learn algorithms, where the uniform error is applied randomly within the
range $[x_i - \sqrt{x_i}, x_i + \sqrt{x_i}]$ for each summed energy window
$x_i$. For the \gls{MLL} calculations, Equation \ref{eq:mllunc} is used, where
$\sigma_{i} = \sqrt{x_i}$.  

In summary, there are four steps taken to arrive at training databases based on
gamma spectra from six chosen detectors. Unlike the simple increase in training
set error applied in Chapter \ref{ch:exp1}, each step of the process adds an
additional layer of reduced information quality, which are listed here:
\begin{enumerate}
  \item The list of nuclides is limited to manually chosen radionuclides.
  \item Instead of "perfect" radionuclide activity knowledge, they are being 
        measured by a gamma detector.
  \item The processing of the gamma spectra can be highly variable.
  \item The $\sqrt{n}$ error of counts-based detection is included. 
\end{enumerate}

