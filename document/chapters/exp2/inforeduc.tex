\begin{figure}[H]
  \centering
  \includegraphics[width=0.7\linewidth]{./chapters/exp2/methodology2_2.png}
  \caption{Second portion of the flowchart from Figure \ref{fig:method2} being 
           described in this section.}
\end{figure}

The overall goal of this project is to determine how much information to what
quality is needed to train an \todo{edit doc for a/an w/ acronyms} \gls{ML}
model that can provide \gls{SNF} attribution by correctly predicting the
reactor type, burnup, \gls{U235} enrichment, and time since irradiation.  In
this section, the information quality is treated as the information reduction
(i.e. decreasing energy resolution) of processed gamma spectra in the training
database.  This detector-based treatment was not applied to the mass
measurements from the first experiment in Chapter \ref{ch:exp1} because
studying measurement techniques that can only be done in a lab is not the goal
of this work.  Instead, field-deployable detectors are of interest.

This process is outlined here for the second experiment, in which a gamma
spectrum is computed for each sample in the database from the nuclide
activities in Section \ref{sec:training2}.  The code \gls{GADRAS} \cite{gadras}
developed at Sandia National Laboratories will provide computational gamma
spectra.  This adds more than one layer of reduced information quality, which
are listed here, and also serve as an outline of the steps taken to process the 
gamma spectra:
\begin{enumerate}
  \item \label{itm:1} The list of nuclides is limited to manually chosen 
        radionuclides.
  \item \label{itm:2} Instead of "perfect" radionuclide activity knowledge, 
        they are being measured by a gamma detector.
  \item \label{itm:3} The processing of the gamma spectra can be highly variable.
  \item \label{itm:4} The $\sqrt{n}$ error of counts-based detection is included. 
\end{enumerate}

\noindent \textbf{Steps 1 \& 2: Choosing Radionuclides \& Computational Gamma Detection}

Step \ref{itm:1} is covered in Section \ref{sec:training2}.  Step \ref{itm:2}
refers to obtaining a gamma spectrum for every \gls{SNF} entry in the database.
This is done using the \gls{GADRAS} tool, which applies a \gls{DRF} to the
gamma lines from these radionuclides. The input is a nuclide activity vector,
and the output is an array of energy bins (measured in $keV$), and the counts
per energy bin.

The nuclide activity data requires some processing to be used in this way.  The
activities that come from the \gls{ORIGEN} simulations are based on there being
$1\:MT$ of initial uranium-based fuel. Not only is this quantity an unlikely
amount to be smuggled, it would overwhelm a detector at the default/calibrated
source-detector distances in \gls{GADRAS}.  Therefore, the material (and
resulting nuclide activities) are scaled to be $1\:g$ of \gls{SNF}.
\todo[inline]{fix this to be better technical explanation of dead time and pile
up} 

Regarding the input information for the \gls{GADRAS} calculations, the sources
are provided without any background; this is because any spectrum would undergo
background subtraction before further analysis. Additionally, the nuclides are
pre-decayed in \gls{ORIGEN} to correspond to various cooling times, but the
source age provided to \gls{GADRAS} needs to be a non-zero value. A source age
of $20\:minutes$ provides the expected peaks.

\begin{table}[!htb]
  \centering
  \begin{tabular}{@{}lcllll@{}}
  \toprule
    \textbf{Detector} &
    \textbf{\begin{tabular}[c]{@{}c@{}}\% FWHM \\ @ 661 keV\end{tabular}} &
    \textbf{\begin{tabular}[c]{@{}l@{}}Distance \\ (cm)\end{tabular}} &
    \textbf{\begin{tabular}[c]{@{}l@{}}Height \\ (cm)\end{tabular}} &
    \textbf{\begin{tabular}[c]{@{}l@{}}Live Time\\ (s)\end{tabular}} &
    \textbf{\begin{tabular}[c]{@{}l@{}}Num \\ Channels\end{tabular}} \\ \midrule
    In-Lab HPGe           & 0.21 & 100.0 & 84.0  & 600  & 8192 \\
    Portable HPGe         & 0.29 & 100.0 & 100.0 & 600  & 8192 \\
    CZT                   & 1.20 & 100.0 & 100.0 & 600  & 1024 \\
    SrI\textsubscript{2}  & 2.94 & 100.0 & 100.0 & 600  & 1024 \\
    LaBr\textsubscript{3} & 3.63 & 213.0 & 84.5  & 2400 & 1024 \\
    NaI                   & 7.74 & 213.0 & 85.4  & 2400 & 1024 \\ \bottomrule
  \end{tabular}
  \caption{Select details of 6 detector setups used to obtain gamma 
           spectra-based training databases.}
  \label{tbl:detsetups}
\end{table}

Training databases were created for the six detectors outlined in Table
\ref{tbl:detsetups}. They were chosen to compare the highest energy resolution
detector, a lab-based \gls{HPGe}, against the rest, in order of decreasing
energy resolution: portable \gls{HPGe}, \gls{CZT}, \gls{SrI2}, \gls{LaBr3}, and
\gls{NaI} detectors. This is displayed in the table by including the \gls{FWHM}
of the $661\:keV$ peak for Cs137. At this point, there are six versions of the
original database for each detector setup, but there is a full gamma spectrum
for each \gls{SNF} entry. It is not computationally prudent to use full gamma
spectra for training and testing, and so these spectra are processed; this is
step \ref{itm:3} from above, and is outlined as follows.

\noindent \textbf{Step 3: Processing Gamma Spectra}

Step \ref{itm:3} covers the steps taken to process the gamma spectrum generated
for each \gls{SNF} entry into a training set where the features are now summed
windows of a range of energy bins.  There are two main design choices here: the
width of the energy windows and the number of energy windows to include. The
energy window width is a value that was manually chosen and does not change for
each detector.  The different energy window widths are listed in Table
\ref{tbl:enwindows}.

\begin{table}[!htb]
  \centering
  \begin{tabular}{@{}lcm{0.7in}m{0.7in}m{0.7in}@{}}
    \toprule
    \multirow{2}{*}{\textbf{Detector}} &
    \multirow{2}{*}{\textbf{\begin{tabular}[c]{@{}l@{}}Energy Window\\ Size {[keV]}\end{tabular}}} &
    \multicolumn{3}{c}{\textbf{Energy Window List Length}} \\ \cmidrule(l){3-5}
                  &    & Auto & Short & Long \\
    \toprule
    In-Lab HPGe   & 2  & 206  & 42    & 151  \\
    Portable HPGe & 3  & 120  & 42    & 151  \\
    CZT           & 8  & 30   & 42    & 151  \\
    SrI2          & 10 & 17   & 42    & 151  \\
    LaBr3         & 12 & 19   & 42    & 151  \\
    NaI           & 12 & 9    & 42    & 151  \\ 
    \bottomrule
  \end{tabular}
  \caption{Energy window sizes and list lengths for 6 detector setups used to 
           process the gamma spectra-based training databases.}
  \label{tbl:enwindows}
\end{table}

Table \ref{tbl:enwindows} lists three energy window list length columns,
\textit{Auto}, \textit{Short}, and \textit{Long}. These correspond to different
processed training sets that have a different number of energy windows
included.  There are two approaches taken: a nuclear physics-based method that
generates an energy window list based on the gamma energies expected to be
detected (short and long), and an automatic peak search of a manually chosen
gamma spectrum in the full gamma spectra training database (auto). 

The first method creates the short and long lists in Table \ref{tbl:enwindows};
the length of these lists are the same for all detectors because they are based
on the gamma energies most likely to be detected, which is independent of the
detector quality. To obtain these lists, the expected number of decays of each
gamma energy is calculated based on the activities of the 32 tracked nuclides
using the Python for Nuclear Engineering toolkit \cite{pyne}.  An arbitrary
minimum number of decays is chosen at $5e8$ decays, and the long list of 151
gamma energies that remain above this cutoff is created. A higher arbitrary
minimum of $5e10$ decays is also chosen; this threshold creates the short list
of 42 gamma energies. 

The short and long lists of gamma energies correspond to the nuclides in Table
\ref{tbl:enlistnucs}. The 12 nuclides listed come from the long list, and the
subset of seven bold nuclides come from the short list.  To determine a "full
knowledge" scenario for the detector-based training sets, two training sets are
also created with the 7- and 12-nuclide activity lists.  It should be noted
that several (three for each reactor type) training set entries were selected
based on sampling evenly throughout the training set parameters, with the
intention that there would be a set of gamma energies comprised from multiple
entries. However, one sample emerged as a superset of the others. This sample
is thus chosen for the second method, discussed next.

\begin{table}[!htb]
  \centering
  \begin{tabular}{@{}|l|l|l|@{}}
    \hline
    \textbf{Am241} & \textbf{Am243} & Cm243          \\ \hline
    Cm244          & Cm245          & \textbf{Cs134} \\ \hline
    \textbf{Cs137} & Eu152          & \textbf{Eu154} \\ \hline
    \textbf{Kr85}  & Pu238          & \textbf{Sb125} \\ \hline
  \end{tabular}
  \caption{Nuclides that are represented by the gamma energy lines in the 
           energy lists. The entire set of 12 nuclides belongs to the long 
           list, and the 7 bold nuclides belong to the short list.}
  \label{tbl:enlistnucs}
\end{table}

The column denoted as \textit{Auto} in Table \ref{tbl:enwindows} is obtained by
a physics-free approach. It is based on a peak search of a spectrum in the full
gamma spectra training database. The previously mentioned sample is selected
for all six detectors, and a peak searching algorithm implemented in python
using the SciPy toolkit \cite{scipy} is applied. Using the peak search on the
six different spectra for the same sample, the resulting number of energy
windows for each detector is in Table \ref{tbl:enwindows}.  

\begin{figure}[!htb]
  \makebox[\textwidth][c]{\includegraphics[width=\linewidth]{./chapters/exp2/energy_window_example.png}}
  \caption{Slice of an example gamma spectrum in one of the training databases
           showing the windows over the gamma energy peaks.}
  \todo[inline]{this is the portable hpge with the auto energy window list, idx 66796 if relevant}
  \label{fig:enwindows}
\end{figure}

After the three energy window lists are created, the full gamma spectra are
processed into three training sets, one for each list.  The energy window width
for each detector is used to sum the binned counts for each list entry; that
is, the counts are summed for each gamma energy in the list $\pm E_{width}$.
This is visualized in Figure \ref{fig:enwindows}, where a portion of a portable
\gls{HPGe} spectrum is shown with the $\pm3\:keV$ windows from the auto energy
window list.  Three training sets are created for each detector, resulting in
18 detector-based processed gamma spectra training sets.

\noindent \textbf{Step 4: Apply Statistical Counting Error}

Lastly, step \ref{itm:4} involves the inclusion of the counting error for the
summed energy windows. This is quite simple, as statistical counting error of
$n$ counts is $\sqrt{n}$.  As in Section \ref{sec:inforeduc1}, this error gets
applied in the same way for the scikit-learn algorithms, where the uniform
error is applied randomly within the range $[x_i - \sqrt{x_i}, x_i +
\sqrt{x_i}]$ for each summed energy window $x_i$. For the \gls{MLL}
calculations, Equation \ref{eq:mllunc} is used, where $\sigma_{i} =
\sqrt{x_i}$.  
