
\begin{figure}[H]
  \centering
  \includegraphics[width=0.7\linewidth]{./chapters/exp2/methodology2_4.png}
  \caption[Fourth portion of the flowchart from Figure \ref{fig:method2}]
          {Fourth portion of the flowchart from Figure \ref{fig:method2} being 
           described in this section.}
\end{figure}

The prediction performance is measured by the balanced accuracy of the reactor
type classification or the absolute and/or relative error of the regression
cases (burnup, \gls{U235} enrichment, cooling time), which were introduced in
Section \ref{sec:testerr}.  These performance metrics for all four prediction
types and all six detectors are compared across the three algorithms used:
\textit{k}-nearest neighbors (denoted in plots as \textit{kNN}), decision trees
(denoted in plots as \textit{Dec Tree} or \textit{DTree}), and \gls{MLL}
calculations.  In all of the results in this section, the statistics being
reported is on all $4.5 \times 10^5$ entries in the training set.

\begin{figure}[!htb]
  \centering
  \includegraphics[width=0.5\linewidth]{./chapters/exp2/exp2_plot_description.png}
  \caption[Demonstration of plot being used to evaluate the results]
          {Demonstration of plot format being used to evaluate the results, 
           shown in order to explain the axes and baseline.}
  \label{fig:detdemo}
\end{figure}

In order to evaluate the prediction performance degradation with decreasing
information quality via detector energy resolution, a basic plot format is
presented, pictured in Figure \ref{fig:detdemo}.  There are three plots for
each prediction parameter, corresponding to the three energy window lists:
auto, short, and long.  The vertical axis is always oriented so that higher
levels indicate better performance, so is plotted as accuracy score or negative
error.  

The horizontal axis is oriented so that reduced information travels rightward.
First, the 29 nuclide mass training set is included for comparison, followed by
the 32 nuclide activity training set, which is intended to represent a full
knowledge scenario of nuclide activities before gamma detection. These are both
the same for all three energy window lists.  The 7 \& 12 nuclide activities are
next, and they represent the full knowledge scenario for the post-gamma
detection short and long energy window lists, respectively.  So if a window is
presenting the short energy window list results, it will include the data point
from the 7 nuclide activity training set; the long energy window list results
present the 12 nuclide activities results. The 12 nuclide activity training set
is used for the auto energy window list plots, but it is not possible to know
which set best represents the auto results since the peak search approach does
not take physics into account and it varies across detectors. All of the
nuclide mass/nuclide activity training sets are predicted with a 1\% training
set error applied, so that there is a non-zero but low error estimate mimicking
near-perfect information.  Last, the six detectors are in order of decreasing
detector energy resolution. Thus, for each prediction parameter, there are
three plots for each energy window list with these nine horizontal axis
categories. 

The last component of Figure \ref{fig:detdemo} is the red horizontal line.  It
represents a minimum acceptable performance, interpreted from the results in
Section \ref{sec:eval1}. It is drawn at the level of the worst performing
algorithm at 20\% training set error.  This is therefore somewhat arbitrary,
but if the detectors cannot predict above this level, it means that a
detector-based training set with only counting error cannot reach the level of
performance of 20\% error in a training set based on an assay of 29 nuclide
masses.

\subsection{Reactor Type Classification}

To judge the degradation of predictions of the algorithms with decreasing
information quality (detailed in Section \ref{sec:inforeduc2}), a plot for each
energy window list (auto, short, and long) is made.  Figure \ref{fig:rxtr}
shows the balanced accuracy of reactor type classification for the previously
described \textit{x}-axis, where a score of $1$ is perfect prediction and a
score of $0$ is random classification. The error bars reflect a 99\% confidence
interval. The red line that indicates a baseline/minimum acceptable performance
is at a balanced accuracy score of 0.84, which is based on the lowest
performance of all three algorithms at 20\% training set error for the 29
nuclide mass training set in Figure \ref{fig:randerrA}. The blue line that
indicates an ideal performance is at 0.97 balanced accuracy, which is based on
the approximate performance of all three algorithms at 5\% training set error
for the 29 nuclide mass training set. 

\begin{figure}[!htb]
  \centering
  \includegraphics[width=\textwidth]{./chapters/exp2/detector_preds_wrt_enlist_BalAcc_rxtr.png}
  \caption{Prediction performance of reactor type as measured by balanced 
           accuracy with respect to decreasing detector energy resolution 
           for three types of processed gamma spectra.}
  \label{fig:rxtr}
\end{figure}

\begin{figure}[!htb]
  \centering
  \includegraphics[width=\textwidth]{./chapters/exp2/confusion_matrix_nucs_acts.png}
  \caption{Confusion matrices of reactor type prediction for each algorithm 
           for different training sets (all at a 1\% error level): 29 nuclide 
           masses, 32 nuclide activities, 12 nuclide activities, and 7 nuclide 
           activities.}
  \label{fig:cm_nucs_acts}
\end{figure}

\begin{figure}[!htb]
  \centering
  \includegraphics[width=0.83\textwidth]{./chapters/exp2/confusion_matrix_6dets_auto.png}
  \caption{Confusion matrices for auto energy window list training sets.}
  \label{fig:cm_auto}
\end{figure}

\begin{figure}[!htb]
  \centering
  \includegraphics[width=0.83\textwidth]{./chapters/exp2/confusion_matrix_6dets_short.png}
  \caption{Confusion matrices for short energy window list training sets.}
  \label{fig:cm_short}
\end{figure}

\begin{figure}[!htb]
  \centering
  \includegraphics[width=0.83\textwidth]{./chapters/exp2/confusion_matrix_6dets_long.png}
  \caption{Confusion matrices for long energy window list training sets.}
  \label{fig:cm_long}
\end{figure}


\label{sec:exp2_rxtr}

\subsection{Regression Results}

In this section, the results of the regression cases (burnup, enrichment, and
time since irradiation) are presented. For each prediction parameter, the plot
format described in Figure \ref{fig:detdemo} is used to show both the
\gls{MAPE} for each of the three energy window lists.  However, while seeing
the performance decrease with all three algorithms on the same plot is helpful
for getting a bigger picture of the results, a more detailed visual is also
helpful. Box plots provide increased statistical detail and a direct comparison
of mean and median error for each data point, which each taken alone can paint
a drastically different picture of the results.  There is one set of boxes for
each algorithm for a given energy window list-based set of detector training
sets.  Because these are box plots, they are not oriented to the "higher is
better" standard of the \textit{y}-axis of the original prediction performance
plots.  There are the same red baselines, but they are now at a positive value
since the \textit{y}-axis is no longer negative.  Lastly, the full knowledge
cases (the 29 nuclide masses, and the 32, 12, and 7 nuclide activities sets)
are not able to be represented on the same scale as the detector training set
results, so they are excluded from the box plot figures. 

\subsubsection{Burnup Regression}

\begin{figure}[!htb]
  \centering
  \includegraphics[width=\textwidth]{./chapters/exp2/detector_preds_wrt_enlist_MAPE_burn.png}
  \caption{Prediction performance of burnup measured by \gls{MAPE} with 
           respect to decreasing detector energy resolution for three types 
           of processed gamma spectra.}
  \label{fig:burn}
\end{figure}

The goal lines and baselines for the burnup plots in Figure \ref{fig:burn} were
all chosen by the performance of algorithms at a reference point of 20\%
training set error for the 29 nuclide mass training set in in Figures
\ref{fig:randerrB} and \ref{fig:randmaeA}.  For Figure \ref{fig:burn}, the
blue line is at -1\% , which corresponds to the \gls{MLL} performance at the
reference point.  The red baseline is at -4\%, which is the lowest performance
of all three algorithms at the reference point.  
%For Figure \ref{fig:burnB},
the blue line is at $-300\:MWd/MTU$, and the red baseline is at
$-1000\:MWd/MTU$.  These also correspond to \gls{MLL} and \textit{k}-nearest
neighbors, respectively, at the reference point in Figure \ref{fig:randmaeA}.
The same lines are present in Figure \ref{fig:burnbox}, but instead are
positive values so the red line is on top.  \todo[inline]{the other two cases
below have arbitrary baselines for a more discriminatory discussion, maybe that
would be useful to do here too. leaving discussion for later}

\begin{figure}[!hp]
  \centering
  \begin{subfigure}[b]{\textwidth}
    \centering
    \includegraphics[width=0.92\textwidth]{./chapters/exp2/abserror_boxplots_auto_burn.png}
    \caption{Burnup prediction error box plots for auto energy windows list.}
    \label{fig:burnboxA}
  \end{subfigure}
  \vskip\baselineskip
  \begin{subfigure}[b]{\textwidth}
    \centering
    \includegraphics[width=0.92\textwidth]{./chapters/exp2/abserror_boxplots_short_burn.png}
    \caption{Burnup prediction error box plots for short energy windows list.}
    \label{fig:burnboxB}
  \end{subfigure}
  \vskip\baselineskip
  \begin{subfigure}[b]{\textwidth}
    \centering
    \includegraphics[width=0.92\textwidth]{./chapters/exp2/abserror_boxplots_long_burn.png}
    \caption{Burnup prediction error box plots for long energy windows list.}
    \label{fig:burnboxC}
  \end{subfigure}
  \caption{Prediction performance of burnup for six detectors as shown by box 
           plots.}
  \label{fig:burnbox}
\end{figure}

\begin{figure}[!hp]
  \centering
  \begin{subfigure}[b]{\textwidth}
    \centering
    \includegraphics[width=0.92\textwidth]{./chapters/exp2/abserror_boxplots_outliers_auto_burn.png}
    \caption{Burnup prediction error box plots for auto energy windows list.}
    \label{fig:burnboxflyA}
  \end{subfigure}
  \vskip\baselineskip
  \begin{subfigure}[b]{\textwidth}
    \centering
    \includegraphics[width=0.92\textwidth]{./chapters/exp2/abserror_boxplots_outliers_short_burn.png}
    \caption{Burnup prediction error box plots for short energy windows list.}
    \label{fig:burnboxflyB}
  \end{subfigure}
  \vskip\baselineskip
  \begin{subfigure}[b]{\textwidth}
    \centering
    \includegraphics[width=0.92\textwidth]{./chapters/exp2/abserror_boxplots_outliers_long_burn.png}
    \caption{Burnup prediction error box plots for long energy windows list.}
    \label{fig:burnboxflyC}
  \end{subfigure}
  \caption{Prediction performance of burnup for six detectors as shown by box 
           plots.}
  \label{fig:burnboxfly}
\end{figure}

\subsubsection{U235 Enrichment Regression}

The goal lines for the enrichment plots in Figure \ref{fig:enri} were both
chosen by the performance of algorithms at a reference point of 20\% training
set error for the 29 nuclide mass training set in in Figures \ref{fig:randerrC}
and \ref{fig:randmaeB}.  Since the results in these two figures were much
better than the detector results, the baselines were chosen somewhat
arbitrarily.  This is partially in order to draw a line through the clustering
of the detector results.  For Figure \ref{fig:enri}, the blue line is at -6\%
, which corresponds to the \textit{k}-nearest neighbors performance at the
reference point.  The red baseline is at -30\%, which is the previously
mentioned arbitrary choice.  For Figure \ref{fig:enri}, the blue line is at
$-0.15\:\% U235$, and the red baseline is at $-0.6\:\% U235$.  The former
corresponds to \textit{k}-nearest neighbors at the reference point in Figure
\ref{fig:randmaeB}, and the latter is another arbitrary choice based on where
the detector results are clustering.  The same lines are present in Figure
\ref{fig:enribox}, but instead are positive values so the red line is on top.

\begin{figure}[!htb]
  \centering
  \includegraphics[width=\textwidth]{./chapters/exp2/detector_preds_wrt_enlist_MAPE_enri.png}
  \caption{Prediction performance of \gls{U235} enrichment measured by 
           \gls{MAPE} with respect to decreasing detector energy resolution 
           for three types of processed gamma spectra.}
  \label{fig:enri}
\end{figure}

Taken as a whole, the data points in Figure \ref{fig:enri} have a distinct
shape, similar to the behavior in Figue \ref{fig:rxtr}.  The three full
knowledge cases have near-perfect enrichment prediction and the six detectors
for all three energy windows lists are nearly flat. For both Figures
\ref{fig:enri}, the blue line is therefore not making any
meaningful discrimination. The almost even performance across detector types 
and algorithms (with exceptions) \todo{left off here}

\begin{figure}[!hp]
  \centering
  \begin{subfigure}[b]{\textwidth}
    \centering
    \includegraphics[width=0.92\textwidth]{./chapters/exp2/abserror_boxplots_auto_enri.png}
    \caption{\gls{U235} enrichment prediction error box plots for auto energy windows list.}
    \label{fig:enriboxA}
  \end{subfigure}
  \vskip\baselineskip
  \begin{subfigure}[b]{\textwidth}
    \centering
    \includegraphics[width=0.92\textwidth]{./chapters/exp2/abserror_boxplots_short_enri.png}
    \caption{\gls{U235} enrichment prediction error box plots for short energy windows list.}
    \label{fig:enriboxB}
  \end{subfigure}
  \vskip\baselineskip
  \begin{subfigure}[b]{\textwidth}
    \centering
    \includegraphics[width=0.92\textwidth]{./chapters/exp2/abserror_boxplots_long_enri.png}
    \caption{\gls{U235} enrichment prediction error box plots for long energy windows list.}
    \label{fig:enriboxC}
  \end{subfigure}
  \caption{Prediction performance of \gls{U235} enrichment for six detectors as 
           shown by box plots.}
  \label{fig:enribox}
\end{figure}

\begin{figure}[!hp]
  \centering
  \begin{subfigure}[b]{\textwidth}
    \centering
    \includegraphics[width=0.92\textwidth]{./chapters/exp2/abserror_boxplots_outliers_auto_enri.png}
    \caption{\gls{U235} enrichment prediction error box plots for auto energy windows list.}
    \label{fig:enriboxflyA}
  \end{subfigure}
  \vskip\baselineskip
  \begin{subfigure}[b]{\textwidth}
    \centering
    \includegraphics[width=0.92\textwidth]{./chapters/exp2/abserror_boxplots_outliers_short_enri.png}
    \caption{\gls{U235} enrichment prediction error box plots for short energy windows list.}
    \label{fig:enriboxflyB}
  \end{subfigure}
  \vskip\baselineskip
  \begin{subfigure}[b]{\textwidth}
    \centering
    \includegraphics[width=0.92\textwidth]{./chapters/exp2/abserror_boxplots_outliers_long_enri.png}
    \caption{\gls{U235} enrichment prediction error box plots for long energy windows list.}
    \label{fig:enriboxflyC}
  \end{subfigure}
  \caption{Prediction performance of \gls{U235} enrichment for six detectors as 
           shown by box plots.}
  \label{fig:enriboxfly}
\end{figure}

\subsubsection{Time Since Irradiation Regression}

\begin{figure}[!htb]
  \centering
  \includegraphics[width=\textwidth]{./chapters/exp2/detector_preds_wrt_enlist_MAPE_cool.png}
  \caption{Prediction performance of time since irradiation measured by 
           \gls{MAPE} with respect to decreasing detector energy resolution 
           for three types of processed gamma spectra.}
  \label{fig:cool}
\end{figure}

The goal lines and baselines for the time since irradiation plots in Figure
\ref{fig:cool} were all chosen by the performance of algorithms at a reference
point of 20\% training set error for the 29 nuclide mass training set in in
Figures \ref{fig:randerrD} and \ref{fig:randmaeC}, with one exception where an
arbitrary choice needed to be made.  The \textit{k}-nearest neighbors results
had to be excluded from consideration since the behavior of this algorithm
degraded drastically at 20\% training set error.  For Figure \ref{fig:cool},
the blue line is at -6\%, which corresponds to the decision trees performance
at the reference point.  The red baseline is at -10\%, which is the previously
mentioned arbitrary choice, since using the \textit{k}-nearest neighbors value
of -30\% would not yeild an interesting discussion.  For Figure
, the blue line is at $-60\:days$, and the red baseline is at
$-120\:days$.  The latter corresponds to decision trees at the reference point
in Figure \ref{fig:randmaeC}. The former was informed by the value of \gls{MLL}
at the reference point ($-40\:days$), but made to be a little lower to better
evaluate the detector results, none of which reach a negative error of that
level.  The same lines are present in Figure \ref{fig:coolbox}, but instead are
positive values so the red line is on top.

\begin{figure}[!hp]
  \centering
  \begin{subfigure}[b]{\textwidth}
    \centering
    \includegraphics[width=0.92\textwidth]{./chapters/exp2/abserror_boxplots_auto_cool.png}
    \caption{Time since irradiation prediction performance box plots for auto energy windows list.}
    \label{fig:coolboxA}
  \end{subfigure}
  \vskip\baselineskip
  \begin{subfigure}[b]{\textwidth}
    \centering
    \includegraphics[width=0.92\textwidth]{./chapters/exp2/abserror_boxplots_short_cool.png}
    \caption{Time since irradiation prediction performance box plots for short energy windows list.}
    \label{fig:coolboxB}
  \end{subfigure}
  \vskip\baselineskip
  \begin{subfigure}[b]{\textwidth}
    \centering
    \includegraphics[width=0.92\textwidth]{./chapters/exp2/abserror_boxplots_long_cool.png}
    \caption{Time since irradiation prediction performance box plots for long energy windows list.}
    \label{fig:coolboxC}
  \end{subfigure}
  \caption{Prediction performance of time since irradiation for six detectors as 
           shown by box plots.}
  \label{fig:coolbox}
\end{figure}

\begin{figure}[!hp]
  \centering
  \begin{subfigure}[b]{\textwidth}
    \centering
    \includegraphics[width=0.92\textwidth]{./chapters/exp2/abserror_boxplots_outliers_auto_cool.png}
    \caption{Time since irradiation prediction performance box plots for auto energy windows list.}
    \label{fig:coolboxflyA}
  \end{subfigure}
  \vskip\baselineskip
  \begin{subfigure}[b]{\textwidth}
    \centering
    \includegraphics[width=0.92\textwidth]{./chapters/exp2/abserror_boxplots_outliers_short_cool.png}
    \caption{Time since irradiation prediction performance box plots for short energy windows list.}
    \label{fig:coolboxflyB}
  \end{subfigure}
  \vskip\baselineskip
  \begin{subfigure}[b]{\textwidth}
    \centering
    \includegraphics[width=0.92\textwidth]{./chapters/exp2/abserror_boxplots_outliers_long_cool.png}
    \caption{Time since irradiation prediction performance box plots for long energy windows list.}
    \label{fig:coolboxflyC}
  \end{subfigure}
  \caption{Prediction performance of time since irradiation for six detectors as 
           shown by box plots.}
  \label{fig:coolboxfly}
\end{figure}


\label{sec:exp2_reg}
