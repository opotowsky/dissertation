
In this section, the results of the regression cases (burnup, enrichment, and
time since irradiation) are presented. For each prediction parameter, the plot
format described in Figure \ref{fig:detdemo} is used to show both the
\gls{MAPE} for each of the three energy window lists.  However, while seeing
the performance decrease with all three algorithms on the same plot is helpful
for getting a bigger picture of the results, a more detailed visual is also
helpful. Box plots provide increased statistical detail and a direct comparison
of mean and median error for each data point, which each taken alone can paint
a drastically different picture of the results.  There is one set of boxes for
each algorithm for a given energy window list-based set of detector training
sets.  Because these are box plots, they are not oriented to the "higher is
better" standard of the \textit{y}-axis of the original prediction performance
plots.  There are the same red baselines, but they are now at a positive value
since the \textit{y}-axis is no longer negative.  Lastly, the full knowledge
cases (the 29 nuclide masses, and the 32, 12, and 7 nuclide activities sets)
are not able to be represented on the same scale as the detector training set
results, so they are excluded from the box plot figures. 

\subsubsection{Burnup Regression}

\begin{figure}[!htb]
  \centering
  \includegraphics[width=\textwidth]{./chapters/exp2/detector_preds_wrt_enlist_MAPE_burn.png}
  \caption{Prediction performance of burnup measured by \gls{MAPE} with 
           respect to decreasing detector energy resolution for three types 
           of processed gamma spectra.}
  \label{fig:burn}
\end{figure}

The goal lines and baselines for the burnup plots in Figure \ref{fig:burn} were
all chosen by the performance of algorithms at a reference point of 20\%
training set error for the 29 nuclide mass training set in in Figures
\ref{fig:randerrB} and \ref{fig:randmaeA}.  For Figure \ref{fig:burn}, the
blue line is at -1\% , which corresponds to the \gls{MLL} performance at the
reference point.  The red baseline is at -4\%, which is the lowest performance
of all three algorithms at the reference point.  
%For Figure \ref{fig:burnB},
the blue line is at $-300\:MWd/MTU$, and the red baseline is at
$-1000\:MWd/MTU$.  These also correspond to \gls{MLL} and \textit{k}-nearest
neighbors, respectively, at the reference point in Figure \ref{fig:randmaeA}.
The same lines are present in Figure \ref{fig:burnbox}, but instead are
positive values so the red line is on top.  \todo[inline]{the other two cases
below have arbitrary baselines for a more discriminatory discussion, maybe that
would be useful to do here too. leaving discussion for later}

\begin{figure}[!hp]
  \centering
  \begin{subfigure}[b]{\textwidth}
    \centering
    \includegraphics[width=0.92\textwidth]{./chapters/exp2/abserror_boxplots_auto_burn.png}
    \caption{Burnup prediction error box plots for auto energy windows list.}
    \label{fig:burnboxA}
  \end{subfigure}
  \vskip\baselineskip
  \begin{subfigure}[b]{\textwidth}
    \centering
    \includegraphics[width=0.92\textwidth]{./chapters/exp2/abserror_boxplots_short_burn.png}
    \caption{Burnup prediction error box plots for short energy windows list.}
    \label{fig:burnboxB}
  \end{subfigure}
  \vskip\baselineskip
  \begin{subfigure}[b]{\textwidth}
    \centering
    \includegraphics[width=0.92\textwidth]{./chapters/exp2/abserror_boxplots_long_burn.png}
    \caption{Burnup prediction error box plots for long energy windows list.}
    \label{fig:burnboxC}
  \end{subfigure}
  \caption{Prediction performance of burnup for six detectors as shown by box 
           plots.}
  \label{fig:burnbox}
\end{figure}

\begin{figure}[!hp]
  \centering
  \begin{subfigure}[b]{\textwidth}
    \centering
    \includegraphics[width=0.92\textwidth]{./chapters/exp2/abserror_boxplots_outliers_auto_burn.png}
    \caption{Burnup prediction error box plots for auto energy windows list.}
    \label{fig:burnboxflyA}
  \end{subfigure}
  \vskip\baselineskip
  \begin{subfigure}[b]{\textwidth}
    \centering
    \includegraphics[width=0.92\textwidth]{./chapters/exp2/abserror_boxplots_outliers_short_burn.png}
    \caption{Burnup prediction error box plots for short energy windows list.}
    \label{fig:burnboxflyB}
  \end{subfigure}
  \vskip\baselineskip
  \begin{subfigure}[b]{\textwidth}
    \centering
    \includegraphics[width=0.92\textwidth]{./chapters/exp2/abserror_boxplots_outliers_long_burn.png}
    \caption{Burnup prediction error box plots for long energy windows list.}
    \label{fig:burnboxflyC}
  \end{subfigure}
  \caption{Prediction performance of burnup for six detectors as shown by box 
           plots.}
  \label{fig:burnboxfly}
\end{figure}

\subsubsection{U235 Enrichment Regression}

The goal lines for the enrichment plots in Figure \ref{fig:enri} were both
chosen by the performance of algorithms at a reference point of 20\% training
set error for the 29 nuclide mass training set in in Figures \ref{fig:randerrC}
and \ref{fig:randmaeB}.  Since the results in these two figures were much
better than the detector results, the baselines were chosen somewhat
arbitrarily.  This is partially in order to draw a line through the clustering
of the detector results.  For Figure \ref{fig:enri}, the blue line is at -6\%
, which corresponds to the \textit{k}-nearest neighbors performance at the
reference point.  The red baseline is at -30\%, which is the previously
mentioned arbitrary choice.  For Figure \ref{fig:enri}, the blue line is at
$-0.15\:\% U235$, and the red baseline is at $-0.6\:\% U235$.  The former
corresponds to \textit{k}-nearest neighbors at the reference point in Figure
\ref{fig:randmaeB}, and the latter is another arbitrary choice based on where
the detector results are clustering.  The same lines are present in Figure
\ref{fig:enribox}, but instead are positive values so the red line is on top.

\begin{figure}[!htb]
  \centering
  \includegraphics[width=\textwidth]{./chapters/exp2/detector_preds_wrt_enlist_MAPE_enri.png}
  \caption{Prediction performance of \gls{U235} enrichment measured by 
           \gls{MAPE} with respect to decreasing detector energy resolution 
           for three types of processed gamma spectra.}
  \label{fig:enri}
\end{figure}

Taken as a whole, the data points in Figure \ref{fig:enri} have a distinct
shape, similar to the behavior in Figue \ref{fig:rxtr}.  The three full
knowledge cases have near-perfect enrichment prediction and the six detectors
for all three energy windows lists are nearly flat. For both Figures
\ref{fig:enri}, the blue line is therefore not making any
meaningful discrimination. The almost even performance across detector types 
and algorithms (with exceptions) \todo{left off here}

\begin{figure}[!hp]
  \centering
  \begin{subfigure}[b]{\textwidth}
    \centering
    \includegraphics[width=0.92\textwidth]{./chapters/exp2/abserror_boxplots_auto_enri.png}
    \caption{\gls{U235} enrichment prediction error box plots for auto energy windows list.}
    \label{fig:enriboxA}
  \end{subfigure}
  \vskip\baselineskip
  \begin{subfigure}[b]{\textwidth}
    \centering
    \includegraphics[width=0.92\textwidth]{./chapters/exp2/abserror_boxplots_short_enri.png}
    \caption{\gls{U235} enrichment prediction error box plots for short energy windows list.}
    \label{fig:enriboxB}
  \end{subfigure}
  \vskip\baselineskip
  \begin{subfigure}[b]{\textwidth}
    \centering
    \includegraphics[width=0.92\textwidth]{./chapters/exp2/abserror_boxplots_long_enri.png}
    \caption{\gls{U235} enrichment prediction error box plots for long energy windows list.}
    \label{fig:enriboxC}
  \end{subfigure}
  \caption{Prediction performance of \gls{U235} enrichment for six detectors as 
           shown by box plots.}
  \label{fig:enribox}
\end{figure}

\begin{figure}[!hp]
  \centering
  \begin{subfigure}[b]{\textwidth}
    \centering
    \includegraphics[width=0.92\textwidth]{./chapters/exp2/abserror_boxplots_outliers_auto_enri.png}
    \caption{\gls{U235} enrichment prediction error box plots for auto energy windows list.}
    \label{fig:enriboxflyA}
  \end{subfigure}
  \vskip\baselineskip
  \begin{subfigure}[b]{\textwidth}
    \centering
    \includegraphics[width=0.92\textwidth]{./chapters/exp2/abserror_boxplots_outliers_short_enri.png}
    \caption{\gls{U235} enrichment prediction error box plots for short energy windows list.}
    \label{fig:enriboxflyB}
  \end{subfigure}
  \vskip\baselineskip
  \begin{subfigure}[b]{\textwidth}
    \centering
    \includegraphics[width=0.92\textwidth]{./chapters/exp2/abserror_boxplots_outliers_long_enri.png}
    \caption{\gls{U235} enrichment prediction error box plots for long energy windows list.}
    \label{fig:enriboxflyC}
  \end{subfigure}
  \caption{Prediction performance of \gls{U235} enrichment for six detectors as 
           shown by box plots.}
  \label{fig:enriboxfly}
\end{figure}

\subsubsection{Time Since Irradiation Regression}

\begin{figure}[!htb]
  \centering
  \includegraphics[width=\textwidth]{./chapters/exp2/detector_preds_wrt_enlist_MAPE_cool.png}
  \caption{Prediction performance of time since irradiation measured by 
           \gls{MAPE} with respect to decreasing detector energy resolution 
           for three types of processed gamma spectra.}
  \label{fig:cool}
\end{figure}

The goal lines and baselines for the time since irradiation plots in Figure
\ref{fig:cool} were all chosen by the performance of algorithms at a reference
point of 20\% training set error for the 29 nuclide mass training set in in
Figures \ref{fig:randerrD} and \ref{fig:randmaeC}, with one exception where an
arbitrary choice needed to be made.  The \textit{k}-nearest neighbors results
had to be excluded from consideration since the behavior of this algorithm
degraded drastically at 20\% training set error.  For Figure \ref{fig:cool},
the blue line is at -6\%, which corresponds to the decision trees performance
at the reference point.  The red baseline is at -10\%, which is the previously
mentioned arbitrary choice, since using the \textit{k}-nearest neighbors value
of -30\% would not yeild an interesting discussion.  For Figure
, the blue line is at $-60\:days$, and the red baseline is at
$-120\:days$.  The latter corresponds to decision trees at the reference point
in Figure \ref{fig:randmaeC}. The former was informed by the value of \gls{MLL}
at the reference point ($-40\:days$), but made to be a little lower to better
evaluate the detector results, none of which reach a negative error of that
level.  The same lines are present in Figure \ref{fig:coolbox}, but instead are
positive values so the red line is on top.

\begin{figure}[!hp]
  \centering
  \begin{subfigure}[b]{\textwidth}
    \centering
    \includegraphics[width=0.92\textwidth]{./chapters/exp2/abserror_boxplots_auto_cool.png}
    \caption{Time since irradiation prediction performance box plots for auto energy windows list.}
    \label{fig:coolboxA}
  \end{subfigure}
  \vskip\baselineskip
  \begin{subfigure}[b]{\textwidth}
    \centering
    \includegraphics[width=0.92\textwidth]{./chapters/exp2/abserror_boxplots_short_cool.png}
    \caption{Time since irradiation prediction performance box plots for short energy windows list.}
    \label{fig:coolboxB}
  \end{subfigure}
  \vskip\baselineskip
  \begin{subfigure}[b]{\textwidth}
    \centering
    \includegraphics[width=0.92\textwidth]{./chapters/exp2/abserror_boxplots_long_cool.png}
    \caption{Time since irradiation prediction performance box plots for long energy windows list.}
    \label{fig:coolboxC}
  \end{subfigure}
  \caption{Prediction performance of time since irradiation for six detectors as 
           shown by box plots.}
  \label{fig:coolbox}
\end{figure}

\begin{figure}[!hp]
  \centering
  \begin{subfigure}[b]{\textwidth}
    \centering
    \includegraphics[width=0.92\textwidth]{./chapters/exp2/abserror_boxplots_outliers_auto_cool.png}
    \caption{Time since irradiation prediction performance box plots for auto energy windows list.}
    \label{fig:coolboxflyA}
  \end{subfigure}
  \vskip\baselineskip
  \begin{subfigure}[b]{\textwidth}
    \centering
    \includegraphics[width=0.92\textwidth]{./chapters/exp2/abserror_boxplots_outliers_short_cool.png}
    \caption{Time since irradiation prediction performance box plots for short energy windows list.}
    \label{fig:coolboxflyB}
  \end{subfigure}
  \vskip\baselineskip
  \begin{subfigure}[b]{\textwidth}
    \centering
    \includegraphics[width=0.92\textwidth]{./chapters/exp2/abserror_boxplots_outliers_long_cool.png}
    \caption{Time since irradiation prediction performance box plots for long energy windows list.}
    \label{fig:coolboxflyC}
  \end{subfigure}
  \caption{Prediction performance of time since irradiation for six detectors as 
           shown by box plots.}
  \label{fig:coolboxfly}
\end{figure}

