\begin{figure}[H]
  \centering
  \includegraphics[width=0.7\linewidth]{./chapters/exp2/methodology2_1.png}
  \caption{First portion of the flowchart from Figure \ref{fig:method2} being 
           described in this section.}
\end{figure}

In the second experiment, activities of radionuclides are necessary to
calculate gamma spectra from these values. The \gls{ORIGEN} simulations and
their inputs are the same as Section \ref{sec:training1}, but the outputs
tracked are a different set of nuclides, measured in $Ci$, or $Curies$.  In
this experiment, the full knowledge scenario is the set of 32 nuclide
activities found in Table \ref{tbl:nucacts}.  The simulation parameters that
are being predicted are the same:
\begin{enumerate}
  \item The classification of the \textbf{reactor type}: \gls{PWR}, \gls{BWR}, 
        or \gls{PHWR}.
  \item The \textbf{burnup}: $MWd/MTU$ (or $GWd/MTU$), mega (or giga) 
        watt-days per metric ton of initial uranium.
  \item The \gls{U235} \textbf{enrichment}: $\%\:{}^{235}\text{U}$. 
  \item The \textbf{time since irradiation}, or cooling time: $days$ (or $years$).
\end{enumerate}

\begin{table}[!htb]
  \centering
  \begin{tabular}{@{}|l|l|l|l|l|l|l|l|@{}}
    \hline
    ${}^{227}\text{Ac}$ & \allbold{${}^{241}\text{Am}$} &
    \allbold{${}^{243}\text{Am}$} & ${}^{133}\text{Ba}$ & ${}^{249}\text{Cf}$ &
    ${}^{252}\text{Cf}$ & ${}^{243}\text{Cm}$ & \allbold{${}^{244}\text{Cm}$} \\ 
    \hline
    ${}^{245}\text{Cm}$ & \allbold{${}^{134}\text{Cs}$} &
    \allbold{${}^{137}\text{Cs}$} & ${}^{152}\text{Eu}$ &
    \allbold{${}^{154}\text{Eu}$} & ${}^{166m}\text{Ho}$ & ${}^{85}\text{Kr}$ &
    ${}^{94}\text{Nb}$ \\ 
    \hline
    ${}^{236}\text{Np}$ & \allbold{${}^{237}\text{Np}$} & ${}^{231}\text{Pa}$ &
    ${}^{146}\text{Pm}$ & ${}^{236}\text{Pu}$ & \allbold{${}^{238}\text{Pu}$} &
    \allbold{${}^{239}\text{Pu}$} & \allbold{${}^{240}\text{Pu}$} \\ 
    \hline
    ${}^{226}\text{Ra}$ & ${}^{125}\text{Sb}$ & ${}^{228}\text{Th}$ &
    ${}^{229}\text{Th}$ & ${}^{232}\text{U}$  & ${}^{233}\text{U}$ &
    \allbold{${}^{234}\text{U}$}  & \allbold{${}^{235}\text{U}$}  \\ 
    \hline
  \end{tabular}
  \caption{Set of features saved for the second experiment, nuclide activities
           measured in $Ci$. The bold nuclide activities overlap with the 
           nuclides in Table \ref{tbl:nucmass}.}
  \label{tbl:nucacts}
\end{table}

The second feature set with 32 nuclide activities listed in Table
\ref{tbl:nucacts} was designed with the following reasons in mind. First,
nuclide activities are the most straightforward units to use for application to
the \gls{DRF} in the \gls{GADRAS} tool for the second experiment. This process
is used to obtain gamma spectra for each \gls{SNF} entry in the database, which
is detailed in \ref{sec:inforeduc2}.  Second, these specific nuclides were
chosen because they remained after four steps of filtering:
\begin{enumerate}
  \item They exist in the 196-long radionuclide list in \gls{GADRAS}.
  \item They have an activity above $1 \times 10^{-7}\:Ci$ (cutoff chosen to 
  filter out nuclides that are unlikely to produce gamma energy peaks).
  \item They have a half-life longer than $1\:year$ (cutoff chosen based on
  maximum time since irradiation of $16\:years$).
  \item They have at least one gamma energy line above $200\:keV$ (cutoff
  chosen based on low-energy gamma energy peaks being difficult to discern in
  some detectors).
\end{enumerate}

% As before, no presumption of normal distrib is needed

%As discussed in Section \ref{sec:snffeats}, the \gls{MLL} method presumes the
%features have a normal distribution. Figure \ref{fig:actshist} shows the
%distribution of each radionuclide in the training set. Compared to the features
%in Figure \ref{fig:nucshist}, there seem to be even fewer features with a
%normal distribution, with Pu239 being perhaps the only feature fitting this
%assumption. The information-reduced training sets covered next have a similar
%profile of distributions to Figure \ref{fig:actshist}, but with a higher
%percentage of exponential distributions.  Again, this should not impact the
%scikit-learn algorithms but may impact the performance of the \gls{MLL}
%calculations. 
%
%\begin{figure}[!htb]
%  \makebox[\textwidth][c]{\includegraphics[width=\linewidth]{./chapters/exp2/histograms_trainset_features_acts.png}}
%  \caption{Histograms of each of the 32 nuclide activities.}
%  \label{fig:actshist}
%\end{figure}

This one initial training set will undergo several transformations to become
six training sets (one for each detector) for each spectra-processing approach
(three for each set of detectors).  This is described next in Section
\ref{sec:inforeduc2}.
