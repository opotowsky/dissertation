The main purpose of this work is to evaluate the utility of statistical methods
as an approach to determine nuclear forensics-relevant quantities as less
information is available. Machine learning algorithms will be used to train
models to provide these values (e.g., reactor type, time since irradiation,
burnup) from the available information. The training data will be simulated
using the ORIGEN tool \todo{cite}, which will provide an array of nuclide
concentrations as the features ($X$) and the parameters of interest ($y$) are
provided from the simulation inputs.  Information reduction will be carried out
using computationally generated gamma spectra; the radionuclide concentrations
from the simulations can be converted into gamma energies, which then undergo a
detector response calculation to represent real gamma spectra as closely as
possible. Machine learning best practices will be used to evaluate the
performance of the chosen algorithms, and inverse problem theory will
be used to provide an interval of confidence in the model predictions.

The necessary background is covered in Chapter \ref{ch:litrev}.  First, an
introduction to the broader field of nuclear forensics is in Section
\ref{sec:nfoverview} to place this work in the context of the technical mission
areas. After that, a short discussion of the field of machine learning, the
algorithms used, and validation methods are in Section \ref{sec:mlback}.
Section \ref{sec:fcsim} includes information about the codes used to generate
the training data, via fuel cycle simulation, detector response function, and
isotope identification of gamma spectra.  Lastly, a review of statistical
methods being used in studies of forensics analysis is covered next in Section
\ref{sec:stats4nf}. 


After the existing work is discussed and the gap that this work will fill is
identified, the experimental components are introduced next in Chapter
\ref{ch:method}. A demonstration of the methodology is presented next in Chapter
\ref{ch:demo}. Following these two chapters, Chapter xx %\ref{ch:proposal} 
describes the thesis research proposal and corresponding hypotheses. 
Finally, future directions and alternative directions are identified in 
Chapter xx. %\ref{ch:future}
