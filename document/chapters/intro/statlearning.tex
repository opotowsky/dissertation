As previously mentioned, there are two main issues that are being addressed for
forensics of \gls{SNF}: database issues and speed of characterization. Many
have begun considering computational approaches to nuclear forensics problems,
such as the INDEPTH tool for inverse depletion and decay analysis
\cite{weber_2006, weber_2010, weber_2011}. This tool uses an iterative
optimization method involving many forward simulations to obtain reactor
parameters of interest given some initial values. 

Another approach utilizes artificial intelligence to solve nuclear forensics
problems, such as implementing searching algorithms for the database comparison
step \cite{gey_search} and machine learning for determining reactor parameters
from \gls{SNF} characteristics \cite{dayman_feasibility_2013, nicolaou_2006,
nicolaou_2009, nicolaou_2014, robel_2009, jones_viz_2014, jones_snf_2014}.  A
variety of statistical and machine learning tools have been used to
characterize spent fuel by predicting categories or labels (e.g., reactor type, fuel
type) as well as predicting values (e.g., burnup, initial enrichment, cooling
time). The former uses classification algorithms and the latter uses regression
algorithms, many of which can be altered to perform both classification and 
regression.

This work evaluates to what degree statistical methods will be able to predict
reactor parameters with respect to the type of training data used.  There is
some promising work discussed in Section \ref{sec:stats4nf} that shows certain
applications of machine learning can provide an additional tool for solving the
forensics problem, both qualitatively (for visualization) and quantitatively
(for prediction).
