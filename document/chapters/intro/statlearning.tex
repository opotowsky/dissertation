As previously mentioned, there are two main issues that are being addressed for
forensics of \gls{SNF}: database issues and speed of characterization.  Many
have begun considering computational techniques developed by nuclear engineers
to calculate the parameters relevant to nuclear forensics analysis.  One
example is the \gls{INDEPTH} tool \cite{weber_2006, weber_2010, weber_2011}.
\gls{INDEPTH} uses an iterative optimization method involving many forward
simulations to obtain reactor parameters of interest given some initial
guesses. 

Another approach utilizes statistical methods to solve nuclear forensics
problems, such as implementing searching algorithms for the database comparison
step \cite{gey_search} or \gls{ML} algorithms for determining reactor
parameters from \gls{SNF} characteristics \cite{dayman_feasibility_2013,
nicolaou_2006, nicolaou_2009, nicolaou_2014, robel_2009, pu_discrimination,
jones_viz_2014, jones_snf_2014, mll_method, mll_sensitivity, mll_validate}.  A
variety of \gls{ML} tools have been used to characterize \gls{SNF} by
predicting \emph{categories} (e.g., reactor type, fuel type) as well as
predicting \emph{values} (e.g., burnup, initial \gls{U235} enrichment, time
since fuel irradiation).  The former uses classification algorithms and the
latter uses regression algorithms, many of which can be altered to perform both
classification and regression. 

Statistical methods have the uniqueness of requiring minimal domain knowledge
via methods or \gls{ML} algorithms that predict the characteristics or values
of interest \cite{dayman_feasibility_2013, pu_discrimination, robel_2009,
nicolaou_2006, nicolaou_2009, nicolaou_2014, jones_snf_2014, jones_viz_2014,
mll_method, mll_sensitivity, mll_validate}.  In the case of most \gls{ML}
algorithms, they first create a black-box (meaning, unable to be
human-interpreted) statistical model using the entries of measurements in a
database. From that, they can predict the reactor parameters of an unknown
sample based on that model.  Having an \gls{ML} model based on a large number
of simulations may also overcome the challenges of missing data, irregular
uncertainty, or lack of information on other fuel cycles.  This logic also
follows for other computational methods using a large number of simulations.
Although they may not involve a reusable model, they also could overcome
missing data, irregular uncertainties, or ignorance of different or
non-commercial fuel cycles.  Also, statistical methods can be employed to
reduce the dimensions in the forensics databases (i.e., reducing the number of
forensics signatures required to be measured), which is another valuable
characteristic.
