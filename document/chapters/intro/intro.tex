\chapter{Introduction}
\label{ch:intro}

The realm of nuclear security involves many parallel efforts in
nonproliferation (verification of treaty compliance, monitoring for smuggling,
proper storage and transportation of nuclear materials), cyber security,
minimizing stocks of weaponizable materials, disaster response training, and
nuclear forensics. All of these efforts have been continually improving, but
there was a gap regarding the ability of the \gls{US} to coordinate and respond
to a nuclear incident, especially with the technical portion of nuclear
forensics: characterization and analysis. After all, the first textbook on the
topic was published in 2005 \cite{nftext_2005}. In 2006, the \gls{US} \gls{DHS}
founded the \gls{NTNFC} within the \gls{DNDO}. The mission of the \gls{NTNFC}
is to establish a robust nuclear forensics capability to attribute radioactive
materials with demonstrable proof.

There are many fields that contribute to the nuclear forensics capability, such
as radiochemical separations, material collection techniques, improving
detector technology, material library development, and identifying forensic
signatures. These needs vary based on whether the material being collected is
post-detonation (e.g., bomb debris) or pre-detonation (e.g., \gls{SNF}).  In
the pre-detonation realm, this project focuses on statistical methods to
identify correlated material characteristics, which can lead to new forensic
signatures. 


\section{Motivation}
\label{sec:motivation}

Nuclear forensics is an important aspect of deterring nuclear terrorism even
though it is not, at first glance, thought to be preventative nuclear security.
The most common defense of the field is that nuclear forensics capability
deters state actors, not terrorist organizations. While it is true that a
strong capability encourages governments to be more active in prevention of
nuclear terrorism, it can also deter the terrorist organizations as well by
increasing their chances of failure. Less destructive success tends to be more
valued than high-risk mass destruction. In addition to influencing governments
and making nuclear terrorism higher risk for organizations, nuclear forensics
can assist in cutting off certain suppliers of nuclear materials or
technologies (e.g., nuclear specialists that are only involved for financial
reasons, access to state suppliers).  Shutting off the sources builds a
concrete barrier to nuclear terrorism.  Therefore, nuclear forensics is
considered impede this form of terrorism in both tangible and abstract ways
\cite{aps_aaas_forensics}.

Following the prevention value of nuclear forensics, it is important to
understand the process of the technical portion of the investigation and how
that can be improved.  In the event of a nuclear incident, such as the
retrieval of stolen \gls{SNM} or the detonation of a dirty bomb, it is
necessary to learn as much as possible about the source of the materials in a
timely manner. In the case of non-detonated \gls{SNM}, knowing the reactor
parameters that produced it can point investigators in the right direction in
order to determine the chain of custody of the interdicted material.  Section
\ref{sec:nfneeds} covers the specific needs of the nuclear forensics community
for \gls{SNF} provenance, and Section \ref{sec:statscontrib} discusses how
alternative computational approaches are useful, with a focus on why
statistical methods in particular are being pursued. 

\subsection{Needs in Nuclear Forensics}

\begin{frame}
  \frametitle{Nuclear Forensics Investigations}
  \begin{minipage}[t]{0.5\textwidth}
    \textbf{Post-detonation}
    \begin{itemize}
      \item Collection: debris, swipe samples
      \item Characterization: rapid analysis of isotope ratios
      \item Goals
      \begin{itemize}
        \item Inverse problem: reconstruct weapon design/yield
        \item Safety: informing disaster response
      \end{itemize}
      \item Data evaluation
    \end{itemize}
  \end{minipage}%
  \pause
  \begin{minipage}[t]{0.5\textwidth}
    \textbf{Pre-detonation}
    \begin{itemize}
      \item Collection: depends on intercepted material
      \item \boxalert{Characterization:} non-destructive and destructive
      \item Goals:
      \begin{itemize}
        \item \boxalert{Inverse problem:} material chain of custody
        \item Safety: material handling and security
      \end{itemize}
      \item \boxalert{Data evaluation}
    \end{itemize}
  \end{minipage}
\end{frame}


\begin{frame}
  \frametitle{Nuclear Forensics as an Inverse Problem}
  Necessary to determine the quality of prediction

  Use Bayes' Framework:
  $$ P(A|B) = \frac{P(B|A)P(A)}{P(B)} $$
  $$ P(M|D) = \frac{P(D|M)P(M)}{P(D)} $$
\end{frame}


\label{sec:nfneeds}

\subsection{Contribution of Statistical Methods}
As previously mentioned, there are two main issues that are being addressed for
forensics of \gls{SNF}: database issues and speed of characterization. Many
have begun considering computational approaches to nuclear forensics problems,
such as the INDEPTH tool for inverse depletion and decay analysis
\cite{weber_2006, weber_2010, weber_2011}. This tool uses an iterative
optimization method involving many forward simulations to obtain reactor
parameters of interest given some initial values. 

Another approach utilizes artificial intelligence to solve nuclear forensics
problems, such as implementing searching algorithms for the database comparison
step \cite{gey_search} and machine learning for determining reactor parameters
from \gls{SNF} characteristics \cite{dayman_feasibility_2013, nicolaou_2006,
nicolaou_2009, nicolaou_2014, robel_2009, jones_viz_2014, jones_snf_2014}.  A
variety of statistical and machine learning tools have been used to
characterize spent fuel by predicting categories or labels (reactor type, fuel
type) as well as predicting values (burnup, initial enrichment, or cooling
time) The former uses classification algorithms and the latter uses regression
algorithms. Many algorithms can be applied to both cases.

A typical (supervised) machine learning workflow would take a set of training
data with labels or values inserted into some statistical learner, calculate
some objective, minimize or maximize that objective, and provide some model
based on that output. Then a test set (with known values) is provided to the
model so that its performance can be evaluated and finalized. After model
finalization, a user can provide a single instance and a value can be predicted
from that. \todo{insert ML schematic}

To obtain reliable models, one must 1. choose/create a training set carefully
and 2. study the impact of various algorithm parameters on the error. Many
algorithms are developed on an assumption that the training set will be
independent and identially distributed (i.i.d.). [Aside: there are ways to
handle skewed data sets] This is important so that the model does not overvalue
or overfit a certain area in the training space. Additionally, algorithm
performance (or error) can be optimized with respect to training set size,
number of features, or algorithm parameters (regularization terms, etc).  These
are known as diagnostic plots. When plotting the training and testing error
with respect to the number of instances, this is known as a learning curve.
When plotting these errors with respect to the number or features or algorithm
parameters, this is known as a validation curve. \todo{insert example
diagnostic plot?}

Algorithm choice is usually based on what is being predicted and intuition
regarding strengths and weaknesses.  For the sake of comparison (i.e. weak
validation), some machine learning approaches here are based on previous work
\cite{dayman_feasibility_2013} while also extending to a more complex model
via an algorithm that is known to handle highly dimensional data sets well.
Thus, this paper investigates three regression algorithms: nearest neighbor,
ridge, and support vectors.


It is first important to determine if statistical methods can
overcome the inherent database deficiencies. Next, the statistical methods must
be considered in such a way as to represent a real-world scenario. Although
mass spectrometry techniques provide extremely accurate isotopic information
for analytical methods, they are time-consuming and more expensive. And
although gamma spectroscopy can give extremely fast results cheaply, it only
measures certain radiological signals and is influenced by many environmental
factors, storage, and self-attenuation. As different machine learning
algorithms and parameters are investigated, this work focuses on probing the
amount of information required to obtain realistic results.

Because creating databases from real measurements to represent reactor
technologies from around the world is impossible, the database in this study
will be created from high-fidelity simulations via ORIGEN irradiation and
depletion \todo{check actual name of code part used}. In the simulation and
statistical learning paradigm, we need to determine how much information to
what quality is needed to train a machine-learned model; the model must give
appropriate predictions of reactor parameters given a set of measurements from
a test sample of interdicted \gls{SNF}. Of interest to an entity trying to
create a weapon is partially irradiated fuel if they have plutonium separations
capabilities or any radioactive substance in the case of a dirty bomb.
Addressing the former, a set of simulations of \gls{SNF} at different burnups
and cooling times will comprise the database.\todo{rewrite to be clearer}

Can the algorithm overcome the deficiencies of gamma detection and still
provide useful results? Or does it need more information, e.g., exact
isotopics? First, we must establish some baseline expectations of reactor
parameter prediction and how different algorithms perform. This work is based
off previous work on the subject \cite{dayman_feasibility_2013} regarding
machine learning performace with respect to information reduction, and expands
upon it by also evaluating a more advanced machine learning algorithm: support
vector regression. 



Below is a more in depth discussion of nuclear forensics and
how machine learning can contribute to this research area. After that, an
experimental design is outlined. Lastly, the results are presented and
discussed. 

Thus, ultimately, the goal is to answer the question \textit{How
does the ability to determine forensic-relevant spent nuclear fuel attributes
degrade as less information is available?}. 

%%%%%



\label{sec:statscontrib}

\section{Methodology}
\label{sec:methodology}

As previously mentioned, the typical workflow of the technical portion of a
forensics investigation is to take measurements of an unknown material and
compare those measurements to databases filled with previously measured
standard materials. As this work focuses on \gls{SNF}, these measurements are
elemental, chemical, and radiological in nature.  Because creating databases
from real measurements to represent \gls{SNF} from reactor technologies from
around the world is impossible, the database in this study will be created from
high-fidelity simulations via \gls{ORIGEN} \cite{origen} within the SCALE code
system \cite{scale} for modeling and simulation. 

In the simulation and statistical learning paradigm, we need to determine how
much information to what quality is needed to train a machine-learned model;
the model must give appropriate predictions of reactor parameters given a set
of measurements from a test sample of interdicted \gls{SNF}. Thus, the space 
that the training set encompasses must be chosen carefully so as to represent 
the typical scenario for stolen \gls{SNF}.

The next step is to choose an algorithm that performs statistical learning.
Statistical learners have varied strengths and weaknesses based on what is
being predicted and how they implement optimization.  Chosen for this study are
simple regression algorithms for burnup prediction: nearest neighbor and ridge
regression.  For comparison, support vector regression is used because it is
known to handle highly dimensional data sets well.  These algorithms are
introduced in Section \ref{sec:algs}.

After the training is complete, the results of each models' predictions must be
evaluated.  Typically, a test set is used to compare against the model created
from the training set.  The testing error can therefore be tabulated with
respect to various specifications such as the training set size, number of
features, or algorithm parameters (regularization terms, etc). These results
are broadly known as diagnostic plots and show if the algorithms' predictions
are due to good performance or bad fitting. 

After the models are evaluated using machine learning best practices, it will
be important to compare them both against each other and against other
computational forensics parameter methods. Thus, a Bayesian approach from the
field of inverse problem theory will be used to give the probability density of
the predictions so that the statistically generated predictions can be
evaluated directly against other solutions, such as optimization-based methods
or direct computations. 

Next, information reduction (within the training and/or testing data sets) must
be investigated to extend this workflow to mimic that of the real world. The
primary example investigated here is the reduction of information quality via
gamma ray detectors, as they can provide fast results.  If an algorithm could
overcome the limitations of gamma detection and still provide useful results,
this would warrant further studies and perhaps be field-applicable.

Thus, ultimately, the goal is to answer the question \textit{How
does the ability to determine forensic-relevant spent nuclear fuel attributes
degrade as less information is available?}. 

\section{Goals}

The main purpose of this work is to evaluate the utility of statistical methods
as an approach to determine nuclear forensics-relevant quantities as less
information is available. Machine learning algorithms will be used to train
models to provide these values (e.g., reactor type, time since irradiation,
burnup) from the available information. The training data will be simulated
using the SCALE 6.2 code suite \cite{scale}, which will provide an array of
nuclide concentrations as the features ($X$) and the parameters of interest
($y$) are provided from the simulation inputs.  Information reduction will be
carried out using computationally generated gamma spectra; the radionuclide
concentrations from the simulations can be converted into gamma energies, which
then undergo a detector response calculation to represent real gamma spectra as
closely as possible. Machine learning best practices will be used to evaluate
the performance of the chosen algorithms, and inverse problem theory will be
used to provide an interval of confidence in the model predictions.

The necessary background is covered in Chapter \ref{ch:litrev}.  First, an
introduction to the broader field of nuclear forensics is in Section
\ref{sec:nfoverview} to place this work in the context of the technical mission
areas. After that, a short discussion of the field of machine learning, the
algorithms used, and validation methods are in Section \ref{sec:mlback}.
Section \ref{sec:fcsim} includes information about the codes used to generate
the training data, via fuel cycle simulation, detector response function, and
isotope identification of gamma spectra.  Lastly, a review of statistical
methods being used in studies of forensics analysis is covered next in Section
\ref{sec:stats4nf}. 

After the existing work is discussed and the gap that this work will fill is
identified, the methodology and a demonstration of the experimental components
is introduced next in Chapter \ref{ch:demo_method}.  Chapter \ref{ch:proposal}
first lists some follow-up tasks from the results of the demonstration. Next it
summarizes the official thesis research proposal and presents the corresponding
hypotheses. Finally, future directions and alternative directions are
identified in Chapter \ref{ch:future}.
