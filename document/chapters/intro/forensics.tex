The process of technical nuclear forensics includes the analysis and
interpretation of nuclear material to determine its history, whether that be
intercepted \gls{SNF}, \gls{UOC}, or the debris from an exploded nuclear
device. After the technical portion is complete, intelligence data can be used
to aid in material attribution; this is the overall goal of nuclear forensics. 

After a nuclear incident, the material or debris is sampled and evaluated
through many techniques that provide the following information: material
structure, chemical and elemental compositions, and radioisotopic compositions
and/or ratios.  These measurements or ratios comprise the forensic signatures
of the sample in question. These signatures can be analyzed with specific
domain knowledge; for example, \gls{UOC} will have trace elements depending on
where it was mined from.  They can also be analyzed against a forensics
database in the case of \gls{SNF}.

Measurement needs and techniques vary vastly depending on the material, as does
the type of signature. This study focuses on non-detonated materials,
specifically, \gls{SNF}. It is important to determine if some intercepted
\gls{SNF} is from an undisclosed reactor or a commercial fuel cycle to
attribute it to an entity or state. This is typically done by obtaining select
chemical and elemental signatures and isotopic ratios, and comparing these
measurements to those in an existing forensics database of reference \gls{SNF}.
The signatures for \gls{SNF} correlate to characteristics that can, in a best
case scenario, point to the exact reactor from which the fuel was intercepted.
The reactor parameters of interest are the reactor type, fuel type and
enrichment at beginning of irradiation, cooling time, burnup \cite{weber_2006,
weber_2010, weber_2011}.

The current and future work of this study are designed based on two primary
needs to bolster the \gls{US} nuclear forensics capability: forensics databases
are imperfect, and our best measurement techniques are not always feasible in
an emergency scenario. It is proposed that using a machine-learned model may be
able to combat these issues; this is introduced next in Section
\ref{sec:statscontrib}. 

\gls{SNF} database imperfection is three-fold. First, forensics databases are
kept by individual countries, and thus do not include a measurements from other
reactor technologies around the world. Additionally, a given database will have
widely varying uncertainty depending where and on what instrument the material
was measured. Lastly, some fields in these databases have missing entries,
which presents issues with matching and characterizing \gls{SNF} based on
interpolating between entries. \todo{cite}

The second broad need within the forensics community is post-incident rapid
characterization. A lofty goal would be to develop methods that provide
instantaneous information, reliable enough to guide an investigation (e.g.,
within 24 hours). In the case of \gls{SNF}, it takes weeks in a lab to measure
isotopes via advanced (cooled detector) gamma spectroscopy and mass
spectrometry equipment. A handheld detector that measures gamma spectra could
provide the fast measurements to calculate isotopic ratios for the
above-mentioned fuel parameters of interest.  However, while this
nondestructive analysis is rapid, it is also difficult to evaluate because of
the presence of overlapping peaks and the fact that uncertainties differ
significantly because of the detector reponse, environment, storage,
electronics, etc. Broadly speaking, gamma spectra give less information at a
higher uncertainty than the near-perfect results of some destructive mass
spectrometry techniques, like TIMS\todo{find MS technique and citation}.

