The process of nuclear forensics includes the analysis and interpretation of
nuclear material to determine its history, whether that be intercepted spent
nuclear fuel, uranium ore concentrate, or the debris from an exploded nuclear
device. After the technical portion is complete, intelligence data can be used
to aid in material attribution; this is the overall goal of nuclear forensics. 

This study focuses on non-detonated materials, specifically, spent nuclear
fuel. It is important to determine if some intercepted material is from a
commercial fuel cycle or if it is meant for weapons production (and where the
material was obtained from). 

\todo{Workflow for determining SNF quantities of interest}: measure material, use
isotope content and/or isotope ratios to determine things like reactor type,
fuel type and enrichment at beginning of irradiation, cooling time, burnup.
(Classification, Characterization, Interpretation (Analysis), Reconstruction
(Attribution) - from the New Nuclear Forensics book) (Char methods to get
isotopic ratios or use S/ML, Interp examples) After the material
characteristics are measured, they are matched in a forensics database that
includes some or all this information for pre-existing/pre-measured SNF. These
databases are kept by individual countries, and a given database will have
widely varying uncertainty depending where the material was measured as well as
missing data in some fields.  Therefore, matching can be difficult.

A lofty goal for the forensics community would be to develop methods that
provide instantaneous information that is reliable enough to guide an
investigation (e.g., within 24 hours). Fast measurements to provide isotopic
ratios to calculate the above-mentioned fuel parameters of interest would
provide this via some form of a handheld detector that measures gamma spectra.
However, while this nondestructive analysis is rapid, it is also difficult to
evaluate because of the presence of overlapping peaks.  Thus, gamma spectra
give less information at a higher uncertainty than the near-perfect results of
some destructive mass spectroscopy techniques, like TIMS\todo{find MS technique
and citation}.  Additionally, within gamma spectroscopy techniques (e.g., field
vs. lab detectors), uncertainties can vary significantly because of the
detector reponse, environment, storage, electronics, etc.  However, using a
well-trained machine-learned model may be able to overcome these inherent
issues with gamma spectra. The current and future work of this study is
designed with this in mind. 
