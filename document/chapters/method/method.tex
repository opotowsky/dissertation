\chapter{Experimental Methodology}
\label{ch:method}

This chapter covers the methodology by introducing each experimental component.
This has been split into three sections, which cover the four steps summarized
in Figure \ref{fig:method}.

\begin{figure}[!ht]
  \centering
  \includegraphics[width=0.7\linewidth]{./chapters/method/methodology.png}
  \caption{Flowchart of the experimental methodology and the way each step is being implemented.}
  \label{fig:method}
\end{figure}

Section \ref{sec:training} discusses how the training data set is obtained
through simulations and computational means. The initial training data
simulated via \gls{ORIGEN} is detailed in Section \ref{sec:snfsim} This
provides a set ofi \gls{SNF} observations with known reactor operation
parameters, i.e., labels that are to be predicted.  Next, the information
reduction step is covered in Section \ref{sec:inforeduc}, where random error is
injected or computational gamma spectra are created via the \gls{GADRAS} tool.
After this, the less-precise data set will be input to a statistical learner
for the next step: training a model.

Section \ref{sec:statmodel} covers the implementation of the algorithms
introduced in \ref{sec:algs}. They use the features and labels in the training
data to formulate a model. \todo[inline]{update needed} 

Lastly, the algorithms must be evaluated for their prediction performance when
given test samples (i.e., a new \gls{SNF} measurement that has no labels
according the to algorithm).  The approach is shown in Section \ref{sec:eval}. 

\section{Training Data Simulation}
\label{sec:training}
\begin{figure}[H]
  \centering
  \includegraphics[width=0.7\linewidth]{./chapters/exp1/methodology1.png}
  \caption{First portion of the flowchart from Figure \ref{fig:method} being 
           described in this section.}
\end{figure}

\todo[inline]{intro, these nucs are chosen because they are radionuclides with
long enough half-lives}

In the second experiment, activities of radionuclides are necessary to
calculate gamma spectra from these values.

\begin{table}[!htb]
  \centering
  \begin{tabular}{@{}|l|l|l|l|l|l|l|l|@{}}
    \toprule
    Ac227&\textbf{Am241}&\textbf{Am243}&Ba133&Cf249          & Cf252          & Cm243          & \textbf{Cm244} \\ \midrule
    Cm245&\textbf{Cs134}&\textbf{Cs137}&Eu152&\textbf{Eu154} & Ho166m         & Kr85           & Nb94           \\ \midrule
    Np236&\textbf{Np237}&Pa231         &Pm146&Pu236          & \textbf{Pu238} & \textbf{Pu239} & \textbf{Pu240} \\ \midrule
    Ra226&Sb125         &Th228         &Th229&U232           & U233           & \textbf{U234}  & \textbf{U235}  \\ \bottomrule
  \end{tabular}
  \caption{Set of features saved for the second experiment, nuclide activities
           measured in $Curies$. The bold nuclide activities overlap with the 
           nuclides in Table \ref{tbl:nucmass}.}
  \label{tbl:nucacts}
\end{table}

The second feature set with the 32 nuclide activities listed in Table
\ref{tbl:nucacts} was designed with the following reasons in mind. First,
nuclide activities are the most straightforward units to use for application to
the \gls{DRF} in the \gls{GADRAS} tool for the second experiment. This process
is used to obtain gamma spectra for each \gls{SNF} entry in the database, which
is detailed in \ref{sec:inforeduc2}.  Second, these specific nuclides were
chosen because they fulfill four steps of filtering:
\begin{enumerate}
  \item They exist in the 196-long radionuclide list in \gls{GADRAS}.
  \item They have an activity above $1e-7\:Ci$ (cutoff chosen to filter out
  nuclides that are unlikely to produce gamma energy peaks).
  \item They have a half-life longer than $1\:year$ (cutoff chosen based on
  maximum cooling time of $16\:years$).
  \item They have at least one gamma energy line above $200\:keV$ (cutoff
  chosen based on low-energy gamma energy peaks being difficult to discern in
  some detectors).
\end{enumerate}



\section{Statistical Learning Implementation}
\label{sec:statmodel}

\begin{figure}[H]
  \centering
  \includegraphics[width=0.7\linewidth]{./chapters/exp2/methodology2_3.png}
  \caption[Third portion of the flowchart from Figure \ref{fig:method2}]
          {Third portion of the flowchart from Figure \ref{fig:method2} being 
           described in this section.}
\end{figure}

The chosen algorithms (\textit{k}-nearest neighbors, decision trees, and
\gls{MLL} calculations) are introduced in Section \ref{sec:algs} and their
implementation details are in Section \ref{sec:statmodel1}.  This section will
therefore only cover the implementation differences from the previous work.
The \gls{MLL} calculations are implemented identically to Chapter
\ref{ch:exp1}.  However, the scikit-learn algorithms in this experiment did
undergo a new round of hyperparameter optimization.

The full list of 22 training sets that undergo training and prediction are as
follows: 29 nuclide masses (from Chapter \ref{ch:exp1}), 32 nuclide activities
(full knowledge scenario for all nuclides, whether or not they are present in a
quantity able to be detected), 7 \& 12 nuclide activities (full knowledge for
short and long energy window lists, respectively), and auto, short, and long
energy window lists applied to the six detectors (lab-based \gls{HPGe},
portable \gls{HPGe}, \gls{CZT}, \gls{SrI2}, \gls{LaBr3}, \gls{NaI}). 

\begin{table}[!htb]
  \centering
  \begin{tabular}{@{}llcll@{}}
    \toprule
    \textbf{\begin{tabular}[c]{@{}l@{}}Training Set\\ Description\end{tabular}} &
    \textbf{\begin{tabular}[c]{@{}l@{}}Prediction\\ Parameter\end{tabular}} &
    \textbf{\begin{tabular}[c]{@{}l@{}}\textit{k} \\ (N neighbors)\end{tabular}} &
    \textbf{\begin{tabular}[c]{@{}l@{}}Max \\ Depth\end{tabular}} &
    \textbf{\begin{tabular}[c]{@{}l@{}}Max \\ Features\end{tabular}} \\ 
    \toprule
    \multirow{4}{*}{\begin{tabular}[c]
    {@{}l@{}}29 \\ Nuclide\\ Masses\end{tabular}}          & Reactor Type & 4 & 56 & 29          \\
                                                           & Burnup       & 1 & 77 & 29          \\
                                                           & Enrichment   & 1 & 73 & 29          \\
                                                           & Cooling Time & 2 & 45 & 29          \\ 
                                                           \hline
    \multirow{4}{*}{\begin{tabular}[c]
    {@{}l@{}}32\\ Nuclide\\ Activities\end{tabular}}       & Reactor Type & 1 & 41 & 32          \\
                                                           & Burnup       & 1 & 49 & 32          \\
                                                           & Enrichment   & 1 & 67 & 32          \\
                                                           & Cooling Time & 7 & 56 & 32          \\
                                                           \hline
    \multirow{4}{*}{\begin{tabular}[c]
    {@{}l@{}}7 or 12\\ Nuclide \\ Activities\end{tabular}} & Reactor Type & 1 & 67 & 7 or 12     \\
                                                           & Burnup       & 1 & 78 & 7 or 12     \\
                                                           & Enrichment   & 1 & 60 & 7 or 12     \\
                                                           & Cooling Time & 4 & 68 & 7 or 12     \\
                                                           \hline
    \multirow{4}{*}{\begin{tabular}[c]
    {@{}l@{}}Energy \\ Windows:\\ Short\end{tabular}}      & Reactor Type & 1 & 62 & 42          \\
                                                           & Burnup       & 1 & 62 & 42          \\
                                                           & Enrichment   & 4 & 64 & 42          \\
                                                           & Cooling Time & 2 & 54 & 42          \\
                                                           \hline
    \multirow{4}{*}{\begin{tabular}[c]
    {@{}l@{}}Energy \\ Windows:\\ Long\end{tabular}}       & Reactor Type & 4 & 62 & 151         \\
                                                           & Burnup       & 1 & 51 & 151         \\
                                                           & Enrichment   & 5 & 73 & 151         \\
                                                           & Cooling Time & 2 & 64 & 151         \\
                                                           \hline
    \multirow{4}{*}{\begin{tabular}[c]
    {@{}l@{}}Energy \\ Windows:\\ Auto\end{tabular}}       & Reactor Type & 2 & 61 & None or 150 \\
                                                           & Burnup       & 1 & 52 & None or 150 \\
                                                           & Enrichment   & 4 & 67 & None or 150 \\
                                                           & Cooling Time & 2 & 58 & None or 150 \\ 
    \bottomrule 
  \end{tabular}
  \caption[Optimized algorithm hyperparameters for all training sets in second 
           experiment]
          {Optimized algorithm hyperparameters; the energy lists took all 
           detectors into account.}
  \label{tbl:exp2hypparam}
\end{table}

Table \ref{tbl:exp2hypparam} lists the hyperparameter optimization results for
the 22 training sets. Because the 29 nuclide mass training set is included in
this chapter for comparison, its optimization results are also listed here.
The number of features for decision trees are not limited because the test runs
for optimization provided highly variable results. The only case where this is
not true is with the auto energy windows list for the lab-based \gls{HPGe} with
a length of 206; the maximum features for this one case are limited to 150.
Instead, optimization was carried out only on the maximum depth for decision
trees with keeping the full length of features.  This is an area that could
undergo deeper exploration than what occurs in this work, since the training
sets with large feature sets can become overfit.

The optimization took place in two rounds, where the first round had a coarser
grid of \textit{k} for \textit{k}-nearest neighbors and maximum depth for
decision trees and the second round had a finer grid of parameters. The 7 \& 12
nuclide activity training sets were optimized separately but contain averages
of the two results for the maximum depth, and the higher value of \textit{k}
when the two did not match. The \textit{k} and maximum depths were averaged
across all six detectors for each energy window list length (short, long, and
auto).  There were fairly consistent results from the short and long lists, but
the variable length of the auto-generated energy windows lists (in Table
\ref{tbl:enwindows}) gave a wider range of ideal hyperparameters.



\section{Performance Evaluation}
\label{sec:eval}

\begin{figure}[H]
  \centering
  \includegraphics[width=0.7\linewidth]{./chapters/method/methodology4.png}
  \caption{Fourth portion of the flowchart from Figure \ref{fig:method} being 
           described in this section.}
\end{figure}

Define types of error, and or link back to section

Define metrics choices, and or link back to section

To obtain reliable models, one must both choose or create a training set
carefully and study the impact of various algorithm parameters on the error.
Although the title of this section suggests final steps of confirming a model's
usefulness for predictions, what follows is more of a diagnostic exercise. 
In practice, these analyses can be used for both purposes.
In practice, validation implies more than just
ensuring the models are properly fit to the data.  Perhaps the training set was
not representative of the actual data space, whereas non-statistical methods do
not rely on the data space for results. To both understand the performance of
the models, the results are then evaluated for over- or under-fitting. 

\gls{ML} algorithms are heavily dependent on the training inputs and algorithm
parameters given to them, such as training set sizes, regularization, number of
features in the training set, optimization parameters, etc.  From the results
shown in Section \ref{sec:statmodel}, it is clear there is room for
improvement.  To evaluate these input and parameter variations, diagnostic
plots show the errors between the predicted burnup values and the actual burnup
values with respect to some variable on the \textit{x}-axis.  As previously
introduced in Section \ref{sec:optvalid}, the prediction errors are compared to
the training error to understand the generalization strength. These two errors
are plotted with respect to training set size (learning curves) and the
algorithm parameters governing model complexity (validation curves) to provide
insight into the model fitness. 

In addition to \gls{ML} best practices, another layer of comparison is added
here.  Because it is difficult to ensure consistently representative testing
data, the accuracy of a learned model should not depend on only one testing
set.  The learned model's accuracy is better estimated by using a validation
set. Here, this is implemented as \textit{k}-fold \gls{CV}, introduced
in Section \ref{sec:selectass}. This work includes both the testing error
(using the testing set described in Section \ref{sec:training}) and $5$-fold
\gls{CV} error. The predetermined testing set will allow for comparison
against the previous work it was obtained from \cite{dayman_feasibility_2013},
but it is assumed that \gls{CV} will provide a better indication of
model performance because the entirety of the training set has also been
tested.  The testing error scenario performs fitting and prediction $n=10$
times and averages the errors of those results.

For a given (randomly chosen) training set size between $15$ and $100\%$ of the
total data set, training and prediction rounds were performed for each. 

What one seeks from a learning curve is for the training and \gls{CV}/testing
curves to approach each other, but also for the magnitude of the error to be
acceptable. As the training set size reaches $100\%$, both the training and
\gls{CV} errors do approach the training error curve.  However, the testing
error here is \textbf{lower} than the training error, and this does not happen
unless there is an issue with the training and/or testing sets. One possible
explanation is that the testing set, while the values were chosen to be between
the data points of the training set, somehow fit the model too well. This is
the danger with systematically choosing a testing set, and why the \gls{CV}
error is used. 




