\begin{figure}[H]
  \centering
  \includegraphics[width=0.7\linewidth]{./chapters/exp1/methodology2.png}
  \caption{Second portion of the flowchart from Figure \ref{fig:method} being 
           described in this section.}
\end{figure}

In this section, the information quality reduction is implemented as increasing
the error of the nuclide mass measurements in the training database. After
this, the statistical models will be evaluated as to how they perform under
increasing error in the nuclide measurements. 

The training database for the first experiment is meant to be a proof of
principle with the developed methodology, and mimic a scenario where there is
"perfect knowledge" of a set of nuclides of interest.  But the overall goal of
this project is to determine how much information to what quality is needed to
train an \gls{ML} model that can provide \gls{SNF} attribution by correctly
predicting the reactor type, burnup, \gls{U235} enrichment, and time since
irradiation.  Therefore error and uncertainty were injected into the nuclide
mass measurements in the training database for the machine learning algorithms
and \gls{MLL} calculations, respectively. 

\subsection{Scikit Algorithms}

For the \textit{k}-nearest neighbor and decision tree algorithms, a uniform
error is applied randomly to each nuclide mass as follows.  For a maximum error
percentage of $E_{max}$, each nuclide mass is peturbed by a random value in the
range: $[100-E_{max},100+E_{max}]$.  This occurs for 12 error levels between
$0\% < E_{max} < 20\%$: 0, 0.3, 0.7, 1, 2, 4, 6, 8, 10, 13, 17, 20.  Therefore
the $0\%$ error case represents full knowledge of nuclide masses, and that
knowledge slowly decreases up to $20\%$. 

\subsection{Maximum Log-Likelihood Calculations}

For the \gls{MLL} calculations, a uniform uncertainty was introduced to each
nuclide mass.  Thus, each nuclide is given an uncertainty of 1, 5, 10, 15, and
20\% via:
\begin{equation}
  \label{eq:mllunc}
  \sigma_{Log L}^2 = \sum_j \left( 
                            \frac{r_{j,test} - r_{j,sim}}{\sigma_{j,sim}^2}
                            \right)^2 
                            (\sigma_{j,sim}^2 + \sigma_{j,test}^2)
\end{equation}
where $r_{j,sim}$ and $r_{j,test}$ are the nuclide measurements for the
simulated/training set samples and the test samples, respectively, and
$\sigma_{j,sim}$ and $\sigma_{j,test}$ are their respective standard deviations
calculated from the four uncertainty levels listed above.  \todo[inline]{review
chosen variables and that they match with what ends up in MLL intro}
