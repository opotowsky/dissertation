
\begin{figure}[H]
  \centering
  \includegraphics[width=0.7\linewidth]{./chapters/exp1/methodology4.png}
  \caption{Fourth portion of the flowchart from Figure \ref{fig:method} being 
           described in this section.}
\end{figure}

As previously introduced in Section \ref{sec:testerr}, the prediction
performance is measured by evaluating the accuracy of the reactor type
classification or the error of the regression cases (burnup, \gls{U235}
enrichment, cooling time).  These performance metrics for all four prediction
types are compared across the three algorithms used: \textit{k}-nearest
neighbors (denoted in plots as \textit{kNN}), decision trees (denoted in plots
as \textit{Dec Tree} or \textit{DTree}), and \gls{MLL} calculations.  

\subsection{Random Error Impacts on Prediction}
\label{sec:randerr}

To judge the degradation of predictions of the algorithms with increasing
nuclide mass measurement error (i.e., reduced information quality, detailed in
Section \ref{sec:inforeduc1}), several plots are made with the introduced error
on the \textit{x}-axis and a prediction performance metric on the
\textit{y}-axis.  The \textit{y}-axis is always oriented so that lower is
poorer performance and higher is better performance. This is why Figures
\ref{fig:randburn}--\ref{fig:randcool} present a negative error on the
\textit{y}-axis. Additionally, the data points on all the plots have a small
$\Delta x$ to show error bars that are otherwise impossible to see.

\subsubsection{Reactor Type Classification}
\label{sec:randerrA}

Figure \ref{fig:randrxtr} shows the balanced accuracy of reactor type
classification, where a score of $1$ is perfect prediction and a score of $0$
is random classification. The error bars reflect a 99\% confidence interval.
While the two scikit-learn algorithms follow a very similar path of decreased
accuracy as the error increases, the \gls{MLL} calculation approach appears to
be more robust to the nuclide mass measurement error.  Another interesting
result is that the \gls{MLL} calculation performs slightly worse for low
errors. If the expected measurement errors of nuclide masses in a training
database or in a test sample can be guaranteed to be better than ~2\%, the
\gls{MLL} calculation is no longer the obvious preferred choice for reactor
type prediction.

\begin{figure}[!htb]
  \centering
  \includegraphics[width=0.5\textwidth]{./chapters/exp1/randerr_compare_nuc29_BalAcc_rxtr.png}
  \caption{Prediction performance of reactor type as measured by balanced 
           accuracy with respect to uniform/random error applied to the nuclide 
           mass measurements in the training set.}
  \label{fig:randrxtr}
\end{figure}

\begin{figure}[!htb]
  \centering
  \includegraphics[width=\textwidth]{./chapters/exp1/confusion_matrix_nuc29_3errs.png}
  \caption{Confusion matrices of reactor type prediction for each algorithm 
           at three training set error levels: 1\%, 10\%, and 20\%, in the 
           top, middle, and bottom panels, respectively.}
  \label{fig:cm_nuc29}
\end{figure}

Although the balanced accuracy score provides more information about
classification performance for an imbalanced data set (the training set is
26.8\% \gls{PWR}, 71.6\% \gls{BWR}, and 1.5\% \gls{PHWR}), it still does not
provide much detail about what is being misclassified. To probe this further,
Figure \ref{fig:cm_nuc29} shows three sets of confusion matrices.  The diagonal
squares are the fraction of true positives for each reactor type, where the
predicted label (\textit{x}-axis) is equal to the true label (\textit{y}-axis).
The off diagonal squares are the fraction of false positives for each reactor
type, where the predicted label is something other than the true label.  These
fractions inside the pixels show the raw numbers of each position normalized to
the number of true labels.  It is ideal to have the diagonal be as close to 1
as possible and the off-diagonal pixels be as close to 0 as possible.  In
addition to the fractions being included in the matrices, the raw count values
are also displayed. To allow the fractions to be rapidly perceived, a colorbar
provides perceptually uniform shading for these true positive and false
positive fractions. 

In the top panel of Figure \ref{fig:cm_nuc29}, the three algorithms are
presented for the 1\% random error case. In Figure \ref{fig:randrxtr}, one can
see these three data points on the plot clustered near the top showing
almost-perfect performance.  (A reminder that the true positive fractions in
the confusion matrices do not map directly to the balanced accuracy score,
which puts more weight on the underrepresented classes.) The confusion matrices
give more dimension to this near-perfect reactor type classification
performance. The majority of the misclassification is in \gls{PWR}s being
classified as \gls{BWR}s: 0.4\% for \textit{k}-nearest neighbors and decision
trees, and 1.6\% for \gls{MLL} calculations. Although, there are also some
\gls{BWR}s that are misclassified as \gls{PWR}s: 0.1\% for \textit{k}-nearest
neighbors and decision trees, and 0.5\% for \gls{MLL} calculations.  There are
zero misclassified \gls{PHWR} cases and zero \gls{LWR} cases misclassified as
\gls{PHWR}; the value of 0.000 to three decimals fraction here represents a
real zero-count, but this is not necessarily the case for the other sets of
confusion matrices.  The \gls{PWR}/\gls{BWR} distinction is known to be a
difficult problem, so the correct \gls{PHWR} classifications are not
particularly notable for this discussion. 

The middle panel of Figure \ref{fig:cm_nuc29} shows confusion matrices for the
three algorithms for the 10\% random error case. In Figure \ref{fig:randrxtr},
one can see these three data points on the plot, where the \gls{MLL} point is
near a balanced accuracy score of 1, and the scikit-learn algorithms both have
score of around 0.93. As with the 1\% error case, the majority of the
misclassification is in \gls{PWR}s being classified as \gls{BWR}s: 11.3\% for
\textit{k}-nearest neighbors, 10.3\% for decision trees, and 1.7\% for
\gls{MLL} calculations.  The \gls{BWR}s are being misclassified as \gls{PWR}s
at the following percentages: 3.4\% for \textit{k}-nearest neighbors, 3.6\% for
decision trees, and 0.5\% for \gls{MLL} calculations. Note how the performance
of the \gls{MLL} calculations are nearly the same for both error levels, which
is shown by the \gls{MLL} line in Figure \ref{fig:randrxtr}. Because of the
normalization, the \gls{LWR}s that are misclassified as \gls{PHWR}s appear to
be zero. However, this does happen, just rarely: 15 \gls{BWR}s are classified
as \gls{PHWR} using decision trees. Also, \textit{k}-nearest neighbors and
decision trees misclassified \gls{PHWR} as an \gls{LWR} 2 times using the
former and 20 times using the latter (no \gls{PHWR} misclassifications happened
using \gls{MLL}).

The bottom panel of Figure \ref{fig:cm_nuc29} shows confusion matrices for the
three algorithms for the 20\% random error case. In Figure \ref{fig:randrxtr},
one can see these three data points on the plot, where the \gls{MLL} point is
near a balanced accuracy score of 0.97, \textit{k}-nearest neighbors is around
0.83, and decision trees is around 0.86. As with the previous two error cases,
the majority of the misclassification is in \gls{PWR}s being classified as
\gls{BWR}s: 24.6\% for \textit{k}-nearest neighbors, 20.3\% for decision trees,
and 5.6\% for \gls{MLL} calculations.  The \gls{BWR}s are being misclassified
as \gls{PWR}s at the following percentages: 7.9\% for \textit{k}-nearest
neighbors, 7.3\% for decision trees, and 0.5\% for \gls{MLL} calculations.
\Gls{PHWR}s are misclassified as an \gls{LWR} 100 times for \textit{k}-nearest
neighbors, 82 times for decision trees, and 0 times for \gls{MLL} calculations.
\Gls{LWR}s were misclassified as \gls{PHWR} 42 times for \textit{k}-nearest
neighbors, 66 times for decision trees, and 0 times for \gls{MLL} calculations.

\subsubsection{Regression Results}
\label{sec:randerrB}

Each set of plots for a given prediction parameter in this section shows both
the relative error (\gls{MAPE}) and the absolute error (\gls{MAE}). In addition
to the \gls{MAE} on the second plot for each regression case, the \gls{MedAE}
is represented as a triangle on the plot as well. These three errors taken
together provide more detailed information about the performance of each
algorithm when faced with training set noise.  As previously mentioned, the
\textit{x}-axis has negative errors, so that higher is always better.  

\begin{figure}[!htb]
  \centering
  \begin{subfigure}[b]{0.48\textwidth}
    \centering
    \includegraphics[width=\textwidth]{./chapters/exp1/randerr_compare_nuc29_MAPE_burn.png}
    \caption{Negative \gls{MAPE} of burnup regression with respect to 
             random error.}
    \label{fig:burnmape}
  \end{subfigure}
  \hfill
  \begin{subfigure}[b]{0.5\textwidth}
    \centering
    \includegraphics[width=\textwidth]{./chapters/exp1/randerr_compare_nuc29_MAE_burn.png}
    \caption{Negative \gls{MAE} of burnup regression with respect to 
             random error.}
    \label{fig:burnmae}
  \end{subfigure}
  \caption{Prediction performance of burnup as measured by relative and 
           absolute errors with respect to uniform/random error applied to the 
           nuclide mass measurements in the training set.}
  \label{fig:randburn}
\end{figure}

Figure \ref{fig:randburn} demonstrates the burnup prediction performance, with
the \gls{MAPE} in Figure \ref{fig:burnmape} and the \gls{MAE} and \gls{MedAE}
in Figure \ref{fig:burnmae}. In these figures, the error bars reflect one
standard deviation of the percentage errors or the absolute errors,
respectively.  As with the reactor type prediction in Figure
\ref{fig:randrxtr}, the \gls{MLL} method is robust to training set error but
performs slightly worse at low error values.  All three methods calculate
burnup with a maximum error of -5\% or $-1000\:MWd/MTU$ at 20\% error in the
training set. The \gls{MedAE}s show a more encouraging picture of the
performance as compared to the \gls{MAE}s. It is interesting that the
scikit-learn algorithms and the \gls{MLL} calculations diverge at 5\% training
set error for \gls{MAPE} and \gls{MAE}, but at 10\% training set error for the
\gls{MedAE}.

\begin{figure}[!htb]
  \centering
  \begin{subfigure}[b]{0.49\textwidth}
    \centering
    \includegraphics[width=\textwidth]{./chapters/exp1/randerr_compare_nuc29_MAPE_enri.png}
    \caption{Negative \gls{MAPE} of \gls{U235} enrichment regression with 
             respect to random error.}
    \label{fig:enrimape}
  \end{subfigure}
  \hfill
  \begin{subfigure}[b]{0.49\textwidth}
    \centering
    \includegraphics[width=\textwidth]{./chapters/exp1/randerr_compare_nuc29_MAE_enri.png}
    \caption{Negative \gls{MAE} of \gls{U235} enrichment regression with 
             respect to random error.}
    \label{fig:enrimae}
  \end{subfigure}
  \caption{Prediction performance of enrichment as measured by relative and 
           absolute errors with respect to uniform/random error applied to the 
           nuclide mass measurements in the training set.}
  \label{fig:randenri}
\end{figure}  

Figure \ref{fig:randenri} demonstrates the \gls{U235} enrichment prediction
performance, with the \gls{MAPE} in Figure \ref{fig:enrimape} and the \gls{MAE}
and \gls{MedAE} in Figure \ref{fig:enrimae}. In these figures, the error bars
reflect one standard deviation of the percentage errors or the absolute errors,
respectively.  Again, the \gls{MLL} method is robust to training set error but
performs slightly worse at low error values.  All three methods calculate
enrichment with a maximum error of -6\% or $-0.17\:\%U235$ at 20\% error in the
training set. Again, the \gls{MedAE}s show a more encouraging picture of the
performance as compared to the \gls{MAE}s. The scikit-learn algorithms and the
\gls{MLL} calculations diverge at 10\% training set error for \gls{MAPE} and
\gls{MAE}, but at 15\% training set error for the \gls{MedAE}.

\begin{figure}[!htb]
  \centering
  \begin{subfigure}[b]{0.49\textwidth}
    \centering
    \includegraphics[width=\textwidth]{./chapters/exp1/randerr_compare_nuc29_MAPE_cool.png}
    \caption{Negative \gls{MAPE} of time since irradiation regression with 
             respect to random error.} 
    \label{fig:coolmape}
  \end{subfigure}
  \hfill
  \begin{subfigure}[b]{0.49\textwidth}
    \centering
    \includegraphics[width=\textwidth]{./chapters/exp1/randerr_compare_nuc29_MAE_cool.png}
    \caption{Negative \gls{MAE} of time since irradiation regression with 
             respect to random error.} 
    \label{fig:coolmae}
  \end{subfigure}
  \caption{Prediction performance of time since irradiation as measured by 
           relative and absolute errors with respect to uniform/random error 
           applied to the nuclide mass measurements in the training set.}
  \label{fig:randcool}
\end{figure}

Last, the time since irradiation prediction performance for the three
algorithms with respect to increasing nuclide mass error is pictured in Figure
\ref{fig:randcool}.  The \gls{MAPE} is shown in Figure \ref{fig:enrimape} and
the \gls{MAE} and \gls{MedAE} are shown in Figure \ref{fig:enrimae}.  The error
bars reflect one standard deviation of the percentage errors or the absolute
errors, respectively.  The \gls{MLL} method is most robust to training set
error, but for this prediction parameter, the behavior of \textit{k}-nearest
neighbors is unique here versus the previous two regression categories.  Time
since irradiation is predicted with a maximum error of -30\% or $-550\:days$ at
20\% error in the training set using the \textit{k}-nearest neighbors
algorithm.  In Figure \ref{fig:coolmape}, there is an inset to show more detail
about the behavior of the decision trees and \gls{MLL} calculations curves, so
if one ignores the clearly anomalous \textit{k}-nearest neighbors performance
and focuses instead on decision trees, those values are -6\% and $-120\:days$.
The \gls{MLL} calculations remain nearly horizontal at approximately 2\%
prediction error for all training set error levels.  The \gls{MedAE}, as usual,
shows a more encouraging picture of the performance as compared to the
\gls{MAE}.

Overall, the robustness to training set error that is present in the \gls{MLL}
calculations applies to all of the prediction parameters. The decision trees
performance is second best, and \textit{k}-nearest neighbors performs the
worst.  In the case of time since irradiation, \textit{k}-nearest neighbors
degrades incredibly fast with introduced training set error.  While it is
difficult to draw a baseline for minimum acceptable behavior on these plots,
these performances can serve as a benchmark for the work presented in Chapter
\ref{ch:exp2}. 


\subsubsection{Model Generalization}
\label{sec:randerrC}

Although a key takeaway from Section \ref{sec:randerr} is that the \gls{MLL}
calculations are the most resilient to introduced error in the training set
features for all four prediction categories, there is another aspect of the
algorithm performance not explicitly shown in those plots: generalization.
\Gls{MLL} does not generalize to unseen data; it provides predictions based on
finding the closest training set entry to the test sample.  It (usually)
outperforms the scikit-learn methods in part because the training set is so
large. This is also true for the \textit{k}-nearest neighbor implementations
where $k=1$ (burnup and cooling time, as seen in Table \ref{tbl:exp1hypparam}).

\begin{figure}[!htb]
  \centering
  \begin{subfigure}[b]{0.495\textwidth}
    \centering
    \includegraphics[width=\textwidth]{./chapters/exp1/learncurve_nuc29_err05_BalAcc_rxtr.png}
    \caption{Balanced accuracy of reactor type classification with respect 
             to training set size.}
    \label{fig:learnsA}
  \end{subfigure}
  \hfill
  \begin{subfigure}[b]{0.485\textwidth}
    \centering
    \includegraphics[width=\textwidth]{./chapters/exp1/learncurve_nuc29_err05_MAPE_burn.png}
    \caption{Negative \gls{MAPE} of burnup regression with respect to 
             to training set size.}
    \label{fig:learnsB}
  \end{subfigure}
  \vskip\baselineskip
  \begin{subfigure}[b]{0.48\textwidth}
    \centering
    \includegraphics[width=\textwidth]{./chapters/exp1/learncurve_nuc29_err05_MAPE_enri.png}
    \caption{Negative \gls{MAPE} of \gls{U235} enrichment regression with 
             respect to training set size.}
    \label{fig:learnsC}
  \end{subfigure}
  \hfill
  \begin{subfigure}[b]{0.50\textwidth}
    \centering
    \includegraphics[width=\textwidth]{./chapters/exp1/learncurve_nuc29_err05_MAPE_cool.png}
    \caption{Negative \gls{MAPE} of time since irradiation regression with 
             respect to training set size.}
    \label{fig:learnsD}
  \end{subfigure}
  \caption[Learning curves for all four prediction parameters]
          {Learning curves for reactor type, burnup, enrichment, and time 
           since irradiation with respect to increasing fraction of the 
           training set, for 5\% training set random error.}
  \label{fig:learns}
\end{figure}

One way to show that an algorithm is generalizing well in comparison to others
is to view the shape of its learning curve (introduced in Section
\ref{sec:complexity}): the prediction performance with respect to training set
size.  It is crucial to have the training sets be identical for each algorithm,
so they were created in advance and the learning curves are constructed
manually rather than relying on the scikit-learn method.  Smaller training sets
were created from the original one by taking 80\%, 60\%, 40\%, and 20\% of the
entries. The training sets are all stratified so that the original fractions of
\gls{BWR}, \gls{PWR}, and \gls{PHWR} are retained. They are also built on top
of one another, so the 20\%-size training set is contained within the 40\%-size
set, and so forth.

Learning curves were constructed for all four prediction categories,
demonstrated in Figure \ref{fig:learns}. As in Figures
\ref{fig:randrxtr}--\ref{fig:randcool} , the vertical axis is always oriented
so that lower is poorer performance and higher is better performance; also, the
error bars reflect a 99\% confidence interval for Figure \ref{fig:learnsA}, and
one standard deviation of the average percentage errors for Figures
\ref{fig:learnsB}--\ref{fig:learnsD}.  These learning curves represent the 5\%
random error case in Figures \ref{fig:randrxtr}--\ref{fig:randcool}, so the
scores/errors in these figures are the data points at the 100\% training set
level in Figure \ref{fig:learns}.  Therefore, the leftmost data in Figure
\ref{fig:learns} will show the \gls{MLL} point being slightly above the
scikit-learn points for the reactor type, burnup, and enrichment predictions,
and the \textit{k}-nearest neighbor point is below the \gls{MLL} and decision
trees points for time since irradiation.  

Figure \ref{fig:learnsA} shows that the balanced accuracy score of reactor type
classification for the \gls{MLL} calculations decreases more at lower training
set size than for the scikit-learn algorithms. Here, the curve crosses below
the \textit{k}-nearest neighbors curve at the lowest training set size of 20\%.
For the burnup \gls{MAPE} in Figure \ref{fig:learnsB} and the enrichment
\gls{MAPE} in Figure \ref{fig:learnsC}, the \gls{MLL} curve crosses below both
of the scikit-learn algorithm curves. This happens between 20\% and 40\%
training set size for burnup, and between 40\% and 60\% training set size for
enrichment.  Lastly, Figure \ref{fig:learnsD} shows a different arrangement,
which is to be expected from the results shown in Figure \ref{fig:randcool},
where the \textit{k}-nearest neighbors performance is significantly worse than
the other two algorithms. Because the \textit{k}-nearest neighbors curve and
error bars are so large, there is an inset showing a closeup of the other two
curves above -6\%.  The decision trees and \gls{MLL} calculations curves now
appear to follow the trend in the burnup and enrichment cases, and the
\gls{MLL} curve crosses under the decision trees curve between 20\% and 40\%
training set size.  

There is a dependence on training set size for all three algorithms in Figure
\ref{fig:learns}. For the most part, the \textit{k}-nearest neighbors and
decision trees curves follow an approximately parallel path, whereas the
\gls{MLL} method shows an increased rate of degradation at low training set
sizes. Since this training set is large enough, i.e., the prediction parameters
were included in small enough steps, that \gls{MLL} has consistent performance
at the larger sizes, there is not a concern in this work about its inability to
generalize. It must be noted, however, that the \gls{MLL} approach requires a
fine grid of simulation parameters in a training database to perform better
than the simple scikit-learn algorithms.

\subsubsection{Reactor Type Prior Knowledge}
\label{sec:randerrD}

There is similar work being done to this work that focuses on similar
prediction categories but in a serial manner, i.e., first determining the
reactor type before moving forward with other predictions \cite{serial_ml}.
This makes a lot of sense intuitively, that knowing the reactor type could
allow a more accurate reporting of the other reactor operation parameters,
rather than trying to predict them while blind to the reactor type, which is
what this work does. Therefore, the change in regression performance from the
reactor type-blind predictions to having prior knowledge of the reactor type is
discussed.

The workflow was repeated for the three regression cases where they were
trained on reactor type-filtered training sets. A 5\% random error was applied
to these training sets, and the 5\% random error full training set was used as
comparison. The errors for each algorithm (\textit{k}-nearest neighbors,
decision trees, and \gls{MLL} calculations) were tallied for each regression
category (burnup, enrichment, and time since irradiation) and within that, for
each reactor type (\gls{PWR}, \gls{BWR}, \gls{PHWR}). Two sets of error were
tracked: whether the reactor type was \textit{known} or \textit{unknown} prior
to prediction.

Table \ref{tbl:knownrxtr} shows the \gls{MAPE}s for each regression category
and within that, for each reactor type (\gls{PWR}, \gls{BWR}, \gls{PHWR}).  The
columns are separated first by the algorithms and second by whether the reactor
type was known or unknown prior to prediction, denoted as \textit{K} and
\textit{U}, respectively. Most of these relative errors are quite low, and
around or under 2\%.  So, e.g., despite burnup from \gls{PHWR}s improving, it
was by 0.61\%, a precision of which may not be of concern. Still, these
performance differences can be looked at in more detail.  

\begin{table}[!htb]
  \centering
  \begin{tabular}{@{}llllllll@{}}
    \toprule
    &  &  \multicolumn{2}{c}{\textit{k}NN} 
    &     \multicolumn{2}{c}{Dec Trees} 
    &     \multicolumn{2}{c}{MLL Calcs} \\ 
    \toprule
    \begin{tabular}[c]{@{}l@{}}Prediction\\ Parameter\end{tabular} &
    \begin{tabular}[c]{@{}l@{}}Reactor \\ Type\end{tabular} &
    K & U &  K & U & K & U \\ \midrule
    \multirow{3}{*}{\begin{tabular}[c]{@{}l@{}}Burnup \\ \% {$[MWd/MTU]$} \end{tabular}}
     & PWR  & 0.60  & 0.66  & 0.54 & 0.75 & 0.24 & 0.25 \\
     & BWR  & 0.88  & 0.90  & 0.60 & 0.66 & 0.40 & 0.40 \\
     & PHWR & 0.66  & 1.27  & 0.14 & 0.54 & 0.28 & 0.28 \\ \hline
    \multirow{3}{*}{\begin{tabular}[c]{@{}l@{}}Enrichment \\ \% {$[\%\:{}^{235}\text{U}]$} \end{tabular}}
     & PWR  & 0.85  & 0.99  & 0.36 & 0.48 & 0.26 & 0.29 \\
     & BWR  & 1.14  & 1.16  & 0.51 & 0.54 & 0.45 & 0.46 \\
     & PHWR & 0.00  & 0.00  & 0.00 & 0.02 & 0.00 & 0.00 \\ \hline
    \multirow{3}{*}{\begin{tabular}[c]{@{}l@{}}Time Since \\ Irradiation \\ \% {$[days]$} \end{tabular}}
     & PWR  & 11.44 & 10.48 & 2.35 & 2.19 & 1.55 & 1.46 \\
     & BWR  & 15.39 & 15.48 & 2.27 & 2.28 & 2.06 & 2.05 \\
     & PHWR & 19.32 & 34.41 & 4.96 & 4.52 & 2.30 & 2.30 \\ \bottomrule
  \end{tabular}
  \caption[\acrshort{MAPE}s for three regression cases comparing known versus 
           unknown reactor type prior knowledge]
          {\acrshort{MAPE}s for the three prediction cases for each algorithm 
           at 5\% training set error. \textit{K} refers to \textit{known} 
           reactor type and \textit{U} refers to \textit{unknown} reactor type 
           prior to regression.}
  \label{tbl:knownrxtr}
\end{table}

To better see these performance differences, the percent change in prediction
\gls{MAE} for each algorithm and reactor type between the reactor type being
known versus unknown prior to prediction was calculated as: $100 \cdot
\frac{MAE_{unknown} - MAE_{known}}{MAE_{unknown}}$.  This was chosen to be
relative to the unknown error since that is the benchmark in this case.  Figure
\ref{fig:knownrxtr} is three heatmaps that show this percent change for each
prediction category, algorithm, and reactor type.  This value is reflected by a
diverging color bar as well as a positive or negative percentage in each
square.  The positive percentages indicate the error decreased/improved from
the unknown reactor type case to the known reactor type case.  The negative
percentages indicate the error increased/worsened from the unknown to the known
case. 

\begin{figure}[!htb]
  \centering
  \includegraphics[width=\textwidth]{./chapters/exp1/rxtr-type_known-unknown_diff_err05.png}
  \caption[Heatmaps for three regression cases comparing known versus 
           unknown reactor type prior knowledge]
          {Heatmaps for the three regression cases showing the percent 
           difference in prediction error between a known reactor type 
           and unknown reactor type using a 5\% training set error.}
  \todo[inline]{would this be better shown with a plot that has little arrows? or leave like this?}
  \label{fig:knownrxtr}
\end{figure}

For burnup prediction, most differences are within $\pm10\%$ except for three
scenarios.  The decision tree algorithm has improved burnup prediction for
\gls{PWR}s by 22.0\% and for \gls{PHWR}s by 71.1\% given a known reactor type.
The \textit{k}-nearest neighbors algorithm has 49.5\% improved burnup
prediction for the \gls{PHWR}. For \gls{U235} enrichment, the \gls{PWR}
predictions improve by approximately 18\% for the scikit-learn algorithms.
Even though the value is within $\pm10\%$, this is the only scenario where
there is an appreciable difference in \gls{MLL} performance. The decision tree
enrichment prediction of \gls{PHWR}s also has a sizeable improvement of
100.0\%.  The time since irradiation predictions for the most part do not show
improvement outside of $\pm10\%$. Of note is some volatile behavior for the
\gls{PHWR} case with the scikit-learn algorithms.  While \textit{k}-nearest
neighbors improves by 29.8\%, the decision tree predictions were worse by
12.5\%.  Since the main concern here is showing how prediction performance
improves with prior reactor type knowledge, this reduction in performance is
odd but not worthy of further investigation.

The improvements in the \gls{PHWR} predictions are not surprising since the
generalization of the scikit-learn algorithms could lead to the unique
\gls{PHWR} cases being ignored, since they are after all only 1.5\% of the
training set.  Another interesting result is that the \gls{BWR} predictions
experience no large changes, which makes sense given that they comprise 72\% of
the training set. Also, the \gls{MLL} predictions are approximately the same,
which is expected because this algorithm does not generalize, and the
prediction comes as a set of labels and is therefore already linked to the
reactor type. Overall, it is important to be aware that the regression labels
coming from a \gls{PHWR} will be unlikely to be optimal results (except for
those from \gls{MLL} calculations).

%\begin{table}[!htb]
%  \centering
%  \begin{tabular}{@{}llllllll@{}}
%    \toprule
%    &  &  \multicolumn{2}{c}{\textit{k}NN} 
%    &     \multicolumn{2}{c}{Dec Trees} 
%    &     \multicolumn{2}{c}{MLL Calcs} \\ 
%    \toprule
%    \begin{tabular}[c]{@{}l@{}}Prediction\\ Parameter\end{tabular} &
%    \begin{tabular}[c]{@{}l@{}}Reactor \\ Type\end{tabular} &
%    K & U &  K & U & K & U \\ \midrule
%    \multirow{3}{*}{\begin{tabular}[c]{@{}l@{}}Burnup \\ \% {$[MWd/MTU]$} \end{tabular}}
%     & PWR  & 0.08  & 0.08  & 0.03 & 0.04 & 0.24 & 0.25 \\ 
%     & BWR  & 0.11  & 0.10  & 0.03 & 0.03 & 0.40 & 0.40 \\ 
%     & PHWR & 0.13  & 0.19  & 0.01 & 0.03 & 0.29 & 0.29 \\ \hline
%    \multirow{3}{*}{\begin{tabular}[c]{@{}l@{}}Enrichment \\ \% {$[\%\:{}^{235}\text{U}]$} \end{tabular}}
%     & PWR  & 0.10  & 0.11  & 0.02 & 0.02 & 0.26 & 0.29 \\ 
%     & BWR  & 0.12  & 0.12  & 0.02 & 0.02 & 0.46 & 0.46 \\ 
%     & PHWR & 0.00  & 0.00  & 0.00 & 0.00 & 0.00 & 0.00 \\ \hline
%    \multirow{3}{*}{\begin{tabular}[c]{@{}l@{}}Time Since \\ Irradiation \\ \% {$[days]$} \end{tabular}}
%     & PWR  & 6.47  & 6.13  & 1.50 & 1.43 & 1.55 & 1.46 \\ 
%     & BWR  & 9.18  & 9.15  & 1.56 & 1.53 & 2.07 & 2.05 \\ 
%     & PHWR & 13.78 & 18.62 & 4.51 & 3.71 & 2.30 & 2.30 \\ \bottomrule
%  \end{tabular}
%  \caption{\gls{MAPE}s for the three prediction cases for each algorithm at 1\% 
%           training set error. \textit{K} refers to \textit{known} reactor 
%           type and \textit{U} refers to \textit{unknown} reactor type prior 
%           to regression.}
%  \label{tbl:knownrxtr}
%\end{table}

%\begin{figure}[!htb]
%  \centering
%  \includegraphics[width=\textwidth]{./chapters/exp1/rxtr-type_known-unknown_diff_err01.png}
%  \caption{Heatmaps for the three regression cases showing the percent 
%           difference in prediction error between a known reactor type 
%           and unknown reactor type.}
%  \label{fig:knownrxtr}
%\end{figure}



\subsection{SFCOMPO Test Set}
\label{sec:sfcompo}

The testing described in Section \ref{sec:randerr} describes the process of
evaluating the methodology with test cases drawn from the training database.
It is also helpful to test the methodology against real assays of \gls{SNF}.
The \gls{SFCOMPO} database was created to allow access to these sorts of
measurements linked to the reactor operation parameters being predicted in this
work \cite{sfcompo, valid_sfco}. The only parameter not part of the
\gls{SFCOMPO} database is the time since irradiation, so that is not predicted
here. 

\begin{figure}[!htb]
  \centering
  \includegraphics[width=0.7\textwidth]{./chapters/exp1/sfcompo_scatter_viz.png}
  \caption[Scatter plot of distribution of \acrshort{SFCOMPO} testing set 
           labels]
          {Scatter plot showing the range of reactor operation parameters in 
           the \acrshort{SFCOMPO} testing set that are being predicted.}
  \label{fig:sfcoscatter}
\end{figure}

The database used in this work is a filtered version of all the entries in the
original database. First, only the nuclide concentration measurements are kept
so that the training set measurements could be converted to the units in
\gls{SFCOMPO}.  The assays in \gls{SFCOMPO} are presented as nuclide
concentrations with the units milligrams per grams of initial uranium, or
$mg/gU_i$. The training set of nuclide measurements in $grams$ is converted to
these concentration units prior to prediction is converted to these
concentration units prior to prediction. Second, only the \gls{PWR}, \gls{BWR},
and \gls{PHWR} entries are retained and all other reactor types are excluded.
Third, uranium-gadolinium fuels are not simulated in the training set and
therefore are also removed from the testing set. Last, duplicate entries for
some measurements exist and the first entry is kept in these cases. 

In all, there are 505 test cases that are able to compare against the training
database.  The number of each reactor type is as follows: 312 \glspl{PWR}, 165
\glspl{BWR}, and 28 \glspl{PHWR}. The space of enrichment and burnup values
is visualized in Figure \ref{fig:sfcoscatter}. These are sufficiently
represented in the training set design, as pictured in Figure
\ref{fig:trainhist}, although the proportions of \gls{PWR} and \gls{BWR} are
approximately opposite to the training set. 

There is one main issue with using \gls{SFCOMPO} as a testing set: missing
nuclide measurements.  The feature set of 29 nuclides in Table
\ref{tbl:nucmass} was chosen based on the frequency of these measurements being
present in the database at an arbitrary level of 100 measurements. This
happened before filtering the uranium-gadolinium fuel, so there are some
nuclide measurements present at under 100 counts.  Each nuclide's frequency in
\gls{SFCOMPO} is listed in Table \ref{tbl:missing}.  While every assay contains
several plutonium measurements and most contain uranium measurements as well,
the remaining nuclides are present at a much lower rate. 

\begin{table}[!htb]
  \centering
  \begin{tabular}{>{\raggedleft}m{0.6in}
                                m{0.4in}|
                  >{\raggedleft}m{0.6in}
                                m{0.4in}|
                  >{\raggedleft}m{0.6in}
                                m{0.4in}}
    \toprule
    \rowcolor[gray]{0.88} ${}^{241}\text{Am}$  & 237 & ${}^{145}\text{Nd}$ & 162 & ${}^{147}\text{Sm}$ & 97  \\  
    \rowcolor[gray]{0.95} ${}^{242m}\text{Am}$ & 110 & ${}^{146}\text{Nd}$ & 139 & ${}^{149}\text{Sm}$ & 97  \\ 
    \rowcolor[gray]{0.88} ${}^{243}\text{Am}$  & 203 & ${}^{148}\text{Nd}$ & 275 & ${}^{150}\text{Sm}$ & 97  \\ 
    \rowcolor[gray]{0.95} ${}^{242}\text{Cm}$  & 214 & ${}^{150}\text{Nd}$ & 121 & ${}^{151}\text{Sm}$ & 97  \\ 
    \rowcolor[gray]{0.88} ${}^{244}\text{Cm}$  & 269 & ${}^{237}\text{Np}$ & 155 & ${}^{152}\text{Sm}$ & 97  \\ 
    \rowcolor[gray]{0.95} ${}^{134}\text{Cs}$  & 113 & ${}^{238}\text{Pu}$ & 369 & ${}^{234}\text{U}$  & 355 \\ 
    \rowcolor[gray]{0.88} ${}^{137}\text{Cs}$  & 185 & ${}^{239}\text{Pu}$ & 505 & ${}^{235}\text{U}$  & 479 \\ 
    \rowcolor[gray]{0.95} ${}^{154}\text{Eu}$  & 100 & ${}^{240}\text{Pu}$ & 505 & ${}^{236}\text{U}$  & 462 \\ 
    \rowcolor[gray]{0.88} ${}^{143}\text{Nd}$  & 162 & ${}^{241}\text{Pu}$ & 504 & ${}^{238}\text{U}$  & 433 \\ 
    \rowcolor[gray]{0.95} ${}^{144}\text{Nd}$  & 113 & ${}^{242}\text{Pu}$ & 505 &       &     \\ \bottomrule
  \end{tabular}
  \caption[Number of assays each nuclide is measured for in \acrshort{SFCOMPO}]
          {Number of assays each nuclide is measured for in the 
           \acrshort{SFCOMPO} database.}
  \label{tbl:missing}
\end{table}

Although some algorithms in theory can handle null values in the testing stage,
scikit-learn does not currently include this capability. The \gls{MLL} method
is designed to handle null values, however. This is done by converting them to
zero and filtering out all zero-valued nuclides during the likelihood
calculations. But there is a technique more commonly applied than imputing
missing values with zero. This involves taking the mean or median of the
existing feature measurements in the testing set and applying that value to the
assays in which it is missing.  The remainder of this section discusses using
the three algorithms to predict the \gls{SFCOMPO} test cases where the nulls
are both imputed using zero and the mean.

\subsubsection{Reactor Type Classification}

Table \ref{tbl:sfcorxtr} presents two metrics for the two missing value
techniques: the accuracy and balanced accuracy scores. The accuracy scores for
both the mean-imputed nulls and zero-imputed nulls test sets are mostly under
0.62, which is the fraction of \gls{PWR} entries.  Therefore, a classifier
could predict \gls{PWR} every time and do better than these accuracy scores.
For the zero-imputed nulls test set predictions using \gls{MLL}, however, the
accuracy of 0.72 does exceed the "majority guess" accuracy of 0.62.  Since
\gls{MLL} calculations filter out null values, it is expected that the scores
will be higher for all prediction categories where \gls{MLL} is being used with
the zero-imputed nulls test set. This expected \gls{MLL} performance also holds
true when looking at the balanced accuracy score.  A balanced accuracy score of
0 denotes random guessing, but it can also be negative if the classifications
are worse than random guessing. The balanced accuracy of 0.63 for the
zero-imputed nulls case is a promising result. The balanced accuracies of
\textit{k}-nearest neighbors and decision trees are all quite low. Also, the
higher accuracies correspond to lower balanced accuracies, and vice versa.
Therefore, further investigation is necessary.

\begin{table}[!htb]
  \centering
  \begin{tabular}{@{}m{1.5in}llllll@{}}
    \toprule
    & \multicolumn{3}{m{2in}}{Accuracy Scores} 
    & \multicolumn{3}{l}{Balanced Accuracy Scores} \\ 
    \toprule
    Null Handling    & kNN   & DTree  & MLL   & kNN   & DTree  & MLL    \\ \midrule
    Mean Imputation  & 0.52  & 0.60   & 0.39  & 0.09  & 0.12   & -0.01  \\
    Zero Imputation  & 0.45  & 0.42   & 0.72  & 0.21  & 0.30   & 0.63   \\ \bottomrule
    \end{tabular}
  \caption[Performance of reactor type classification of \acrshort{SFCOMPO} 
           entries]
          {Accuracy and balanced accuracy scores for reactor type prediction 
           of the \acrshort{SFCOMPO} test cases.}
  \label{tbl:sfcorxtr}
\end{table}

\begin{figure}[!htb]
  \centering
  \includegraphics[width=\textwidth]{./chapters/exp1/confusion_matrix_sfco.png}
  \caption[Confusion matrices of reactor type classification of 
           \acrshort{SFCOMPO} entries]
          {Confusion matrices of reactor type prediction for each algorithm 
           using two missing entry techniques: imputation with mean values (top 
           panel) and with zero (bottom panel).}
  \label{fig:cm}
\end{figure}

Figure \ref{fig:cm} allows for a deeper look into what is happening with the
reactor type predictions for both of the \gls{SFCOMPO} test sets by studying
the confusion matrices, which are introduced in Sections \ref{sec:testerr} and
\ref{sec:randerrA}.  The matrices from using mean-imputed null values are in
the top panel and the matrices from using zero-imputed null values are in the
bottom panel.  The scikit-learn algorithms have a higher accuracy for the
mean-imputed test set than their zero-imputed counterparts, but the balanced
accuracies follow the opposite direction. For \textit{k}-nearest neighbors,
mean-imputed nulls cause more than half of the \gls{PWR}s and all of the
\gls{PHWR}s to be misclassified as \gls{BWR}s. Additionally, 28.5\% \gls{BWR}s
are misclassified as \gls{PWR}s.  For the zero-imputed nulls test set, there is
a much larger correct \gls{BWR} classification percentage, but also a much
larger \gls{PWR} misclassification percentage. The \gls{PHWR} true positive
percentage increases from 0 to 25\%.  \Gls{BWR} (32\% of test cases) and
\gls{PHWR} (5.5\% of test cases) are the two minority classes in the database.
Because they both have higher true positive fractions but there were overall
fewer correct predictions (from a much higher \gls{PWR} misclassification) from
the mean-imputed nulls to the zero-imputed nulls, the accuracy decreased but
the balanced accuracy increased.

Decision trees follows a similar pattern the the \textit{k}-nearest neighbors
example.  The \gls{PWR} misclassification increases from the mean-imputed nulls
to the zero-imputed nulls test set in a similar manner, although the original
\gls{PWR} true positive fraction is higher (leading to a higher accuracy for
the mean-imputed nulls case).  As before, the \gls{BWR} correct classification
increases drastically from the mean-imputed nulls to zero-imputed nulls case.
Additionally, the \gls{PHWR} classification improvement from mean-imputed nulls
to zero-imputed nulls is better for decision trees than for \textit{k}-nearest
neighbors.  Therefore, again, the accuracy decreased and the balanced accuracy
increased.  The larger improvement for the minority classes has led to the
larger balanced accuracy improvement for decision trees.

In Figure \ref{fig:cm}, the confusion matrices for the \gls{MLL} calculations
tell a very different story than those for the scikit-learn algorithms.  The
only similarity is that using mean-imputed nulls causes all \gls{PHWR}s to be
classified as \gls{BWR}s. \gls{PWR}s are misclassified as \gls{BWR}s at 78.2\%,
and \gls{BWR}s are misclassified as \gls{PWR}s at 22.4\%. For the zero-imputed
nulls test set, \gls{PHWR}s and \gls{PWR}s true positive percentages sharply
improve to 92.9\% and 79.5\%, respectively.  The true positive rate for
\gls{BWR}s, however, decreases from 77.0\% to 53.9\%. This is the opposite
trend from both of the scikit-learn algorithms.  Despite the misclassification
increase for \gls{BWR}s, both the accuracy score and balanced accuracy score
increase for \gls{MLL} calculations when moving from using mean-imputed null
values to zero-imputed null values in the test set. The improvement in
\gls{MLL} classification is likely because the mean-imputed nulls test set
hides information rather than removing it from consideration, which is what the
zero-imputed nulls test set does. 

All three algorithms using both test sets tend towards misclassifying
\gls{PHWR}s as \gls{BWR}s (except for \gls{MLL} calculations using zero-imputed
null missing values).  This is likely because \gls{BWR}s comprise the majority
of the training set (72\%), and no matter how the missing measurements are
handled there may be too little information to predict these well with most
algorithms.  For the two scikit-learn algorithms, the zero-imputed nulls test
set predicts \gls{BWR}s the majority of the time (despite there being 50\%
correct \gls{PHWR} prediction for decision trees). This also is likely from
\gls{BWR} being the training set majority class, so with many nuclides
measuring at zero, there are likely to be few good matches, and the majority
class becomes the most likely prediction. This explanation is possibly
applicable to the mean-imputed nulls test set as well, but the behavior pattern
is less clear because the mean-imputed nulls give more information for these
algorithms than zero-imputed nulls (since the zero-imputed values cannot be
removed from consideration), so a larger proportion of \gls{PWR}s are being
predicted properly.  

\subsubsection{Regression Cases}
\label{sec:sfcoreg}

Next, the prediction of \gls{SFCOMPO} test samples for the regression cases
will be discussed. There is no time since irradiation value in the database, so
only burnup and \gls{U235} enrichment are discussed here.  While the mean and
median errors for burnup and enrichment prediction are listed in Tables
\ref{tbl:sfcoburn} and \ref{tbl:sfcoenri}, respectively, there are also box
plots included for both absolute and relative errors in Figures
\ref{fig:sfcoburn} and \ref{fig:sfcoenri}, respectively.  Box plots were chosen
since they can provide a larger amount of information than just a mean or
median value.  The white triangles represent the mean error, and the white line
in notched box is the median error. The box itself is the 25\% (Q1) and 75\%
(Q3) quartiles at the bottom and top, respectively. The error bars or whiskers
are meant to represent the spread of all errors minus the outliers.  The bottom
whisker reaches to $Q1 - 1.5(Q3-Q1)$ and the top whisker reaches to $Q3 +
1.5(Q3-Q1)$. Any values outside of the total whisker range, $4(Q3-Q1)$, are
considered outliers. \cite{matplotlib} Lastly, there are plots of the true
value versus the predicted value for both burnup and enrichment in Figures
\ref{fig:yvy_sfcoburn} and \ref{fig:yvy_sfcoenri}, respectively.

\noindent \textbf{Burnup}

The expected results that \gls{MLL} will perform better with the zero-imputed
nulls test set also holds true for the regression cases, as shown in Table
\ref{tbl:sfcoburn}.  While the scikit-learn algorithms have a moderate increase
in both the mean and median burnup errors from the mean-imputed nulls to the
zero-imputed nulls, the \gls{MLL} calculations have an order of magnitude
decrease in error when moving in that same direction.  Seeing this trend
between Figures \ref{fig:burnimp} and \ref{fig:burn0} is a little difficult due
to the different ranges on the vertical axes, but the drastic improvement in
\gls{MLL} burnup error from the mean-imputed nulls in Figure \ref{fig:burnimp}
to the zero-imputed nulls in Figure \ref{fig:burn0} is still visually clear. In
the latter figure, there is one outlier for \textit{k}-nearest neighbors and 39
for \gls{MLL} calculations.

\begin{table}[!htb]
  \centering
  \begin{tabular}{@{}m{1.5in}llllll@{}}
    \toprule
                     & \multicolumn{3}{m{2in}}{Mean Errors $[GWd/MTU]$} 
                     & \multicolumn{3}{c}{Median Errors $[GWd/MTU]$} 
                     \\ \toprule
    Null Handling    & kNN   & DTree & MLL   & kNN   & DTree & MLL    \\ \midrule
    Mean Imputation  & 9.43  & 10.89 & 13.17 & 7.26  & 8.28  & 10.84  \\
    Zero Imputation  & 14.88 & 15.18 & 3.53  & 11.47 & 8.79  & 1.70   \\ \bottomrule
  \end{tabular}
  \caption[Performance of burnup regression of \acrshort{SFCOMPO} entries]
          {Mean and median errors for burnup prediction of the 
          \acrshort{SFCOMPO} test cases.}
  \label{tbl:sfcoburn}
\end{table}

Although the mean and median errors are contained in the range of 
$1-15\:GWd/MTU$, the large spread in burnup errors in Figures \ref{fig:burnimp}
and \ref{fig:burn0} for all three algorithms was broad enough to warrant an
investigation into the range of relative errors, expressed as percent errors in
Figures \ref{fig:burnimppct} and \ref{fig:burn0pct}.  In Figure
\ref{fig:burnimppct} there are 75, 72, and 73 outliers (all around 15\% of the
test database) for the \textit{k}-nearest neighbors, decision trees, and
\gls{MLL} calculations, respectively.  In Figure \ref{fig:burn0pct} there are
45 outliers for the \gls{MLL} calculations.

\begin{figure}[!htb]
  \centering
  \begin{subfigure}[b]{0.49\textwidth}
    \centering
    \includegraphics[width=\textwidth]{./chapters/exp1/sfcompo_boxplots_impnull_burn.png}
    \caption{Box plots of burnup errors using mean-imputed null values.}
    \label{fig:burnimp}
  \end{subfigure}
  \hfill
  \begin{subfigure}[b]{0.49\textwidth}
    \centering
    \includegraphics[width=\textwidth]{./chapters/exp1/sfcompo_boxplots_impnull_pcterr_burn.png}
    \caption{Box plots of burnup percentage errors using mean-imputed null values.}
    \label{fig:burnimppct}
  \end{subfigure}
  \vskip\baselineskip
  \begin{subfigure}[b]{0.49\textwidth}
    \centering
    \includegraphics[width=\textwidth]{./chapters/exp1/sfcompo_boxplots_0null_burn.png}
    \caption{Box plots of burnup errors using zero-imputed null values.}
    \label{fig:burn0}
  \end{subfigure}
  \hfill
  \begin{subfigure}[b]{0.49\textwidth}
    \centering
    \includegraphics[width=\textwidth]{./chapters/exp1/sfcompo_boxplots_0null_pcterr_burn.png}
    \caption{Box plots of burnup percentage errors using zero-imputed null values.}
    \label{fig:burn0pct}
  \end{subfigure}
  \caption[Box plots of burnup regression of \acrshort{SFCOMPO} entries]
          {Box plots of burnup prediction errors and percentage errors for each 
           algorithm using two missing entry techniques: imputation with mean 
           values and with zero.}
  \label{fig:sfcoburn}
\end{figure}

The large ranges seen for the mean-imputed nulls test set in Figure
\ref{fig:burnimppct} are because of large overpredictions of low burnups ($<
10\:GWd/MTU$), since a small number in the denominator will yield a high
percentage error. A large subset of the low burnup cases are \gls{PHWR}s.
Their burnups are also unlikely to be predicted well because of the inability
of \gls{PHWR} reactors to be represented accurately with this methodology.
Removing the \gls{PHWR} reactors from the results removes all outliers with
percentage errors larger than 1750\%. That is still a very high relative error,
but there are other low burnup cases in the database.  Removing \gls{PHWR}s
does not significantly alter the zero-imputed nulls results in Figure
\ref{fig:burn0pct}.

The range of percentage errors for the zero-imputed nulls test set in Figure
\ref{fig:burn0pct} tells a different story. There is only one case (an
\gls{MLL} outlier) that is above 100\% error.  While the absolute errors for
the scikit-learn algorithms in Figure \ref{fig:burn0} span a larger range than
their counterparts in Figure \ref{fig:burnimp}, their relative errors remain
within $0-100\%$. The only case that predicts the burnup well is the \gls{MLL}
method with the zero-imputed missing values treatment of the \gls{SFCOMPO} test
set, but about 8\% of the test cases are outliers.  If the best-case median
error of $1.7\:GWd/MTU$ in Table \ref{tbl:sfcoburn} were to also correspond to
a lower relative error, then that would be an acceptable result.  However,
while the \gls{MLL} calculations have a percentage error below 20\% at the 75\%
quartile, the non-outlier data reaches almost 40\% and the 8\% of the data
reaches 100\%.

\begin{figure}[!htb]
  \centering
  \begin{subfigure}[b]{\textwidth}
    \centering
    \includegraphics[width=\textwidth]{./chapters/exp1/sfcompo_truey_vs_predy_impnull__burn.png}
    \caption{True versus predicted burnup using mean-imputed null values.}
    \label{fig:yvy_burnimp}
  \end{subfigure}
  \vskip\baselineskip
  \begin{subfigure}[b]{\textwidth}
    \centering
    \includegraphics[width=\textwidth]{./chapters/exp1/sfcompo_truey_vs_predy_0null__burn.png}
    \caption{True versus predicted burnup using zero-imputed null values.}
    \label{fig:yvy_burn0}
  \end{subfigure}
  \caption[True versus predicted burnup of \acrshort{SFCOMPO} test cases]
          {True versus predicted burnup for each algorithm using two missing 
           entry techniques for \acrshort{SFCOMPO}: imputation with mean 
           values and with zero.}
  \label{fig:yvy_sfcoburn}
\end{figure}

To better understand the high errors in Figure \ref{fig:sfcoburn} and
especially the relative error outliers, Figure \ref{fig:yvy_sfcoburn} presents
the plots of the true burnup versus the predicted burnup for each test sample
in \gls{SFCOMPO} for each algorithm and both imputation techniques.  The
colorbar is the percentage error and was chosen with the range up to 200\%.
This is in order to show the difference between the large errors below the
diagonal line generally having a maximum error of 100\%, whereas the
mispredicted low-burnup cases have errors exceeding 200\%.  First, Figure
\ref{fig:yvy_burnimp} shows that all of the large errors are centered around a
certain predicted burnup range, $30-40\:GWd/MTU$. This is happening because
mean imputation technique is being applied to a data set where the measurements
likely only exist for a small range of burnup values, and each test sample is
likely to have many missing values.  The very large relative errors from Figure
\ref{fig:burnimppct} ($200\%-5000\%$) are clustered in the same place for all
three algorithms.  For Figure \ref{fig:yvy_burn0}, the large-error predictions
are shown clustered towards the bottom. This makes sense for the scikit-learn
algorithms since the zero measurements are easily interpreted as low burnup. Of
course, this is not the case for the \gls{MLL} calculations since the zero
values are filtered. 

Overall, the absolute errors in Table \ref{tbl:sfcoburn} tell a much more
encouraging story than the box plots in Figure \ref{fig:sfcoburn}, so
investigating beyond the mean and median absolute errors was necessary to show
the real picture of this unique testing scenario. The additional details
provided by directly plotting the true versus predicted burnup in Figure
\ref{fig:yvy_sfcoburn} is crucial in understanding how the null-value handling
methods impacted the results.

\noindent \textbf{\gls{U235} Enrichment}

For both reactor type classification and burnup regression, the zero-imputed
nulls test set predicted by \gls{MLL} calculations far outperform all other
algorithm/test set scenarios. However, the enrichment regression results break
this trend.  Table \ref{tbl:sfcoenri} shows that decision trees outperform the
other methods, and furthermore, there isn't a large difference in performance
between the two test sets, especially seeing that the median absolute error is
the same for both test sets.  The other two algorithms follow their previous
behavior: moving from mean-imputed nulls to zero-imputed nulls,
\textit{k}-nearest neighbors has worse performance and \gls{MLL} calculations
has better performance.

\begin{table}[!htb]
  \centering
  \begin{tabular}{@{}m{1.5in}llllll@{}}
    \toprule
                     & \multicolumn{3}{m{2in}}{Mean Errors [$\%\:{}^{235}\text{U}$]} 
                     & \multicolumn{3}{l}{Median Errors [$\%\:{}^{235}\text{U}$]} 
                     \\ \toprule
    Null Handling    & kNN  & DTree & MLL  & kNN   & DTree & MLL    \\ \midrule
    Mean Imputation  & 0.72 & 0.31  & 1.25 & 0.50  & 0.22  & 1.13   \\
    Zero Imputation  & 1.67 & 0.36  & 0.49 & 2.02  & 0.22  & 0.35   \\ \bottomrule
  \end{tabular}
  \caption[Performance of enrichment regression of \acrshort{SFCOMPO} entries]
          {Mean and median errors for enrichment prediction of the \gls{SFCOMPO} 
           test cases.}
  \label{tbl:sfcoenri}
\end{table}

The mean and median absolute errors are also visible with more statistical
information in the box plots in Figure \ref{fig:sfcoenri}. The outliers for
decision trees and \gls{MLL} calculations are 30 and 16 for the mean-imputed nulls
in Figure \ref{fig:enriimp}, respectively. So although decision trees provides
a typically low absolute error, the outliers reach nearly as far as the spread
of \textit{k}-nearest neighbors. The number of outliers for the zero-imputed nulls
results in Figure \ref{fig:enri0} are 45 and 16 for decision trees and
\gls{MLL} calculations, respectively.  The spread of the outliers for these
algorithms is similar to the previous figure, but \textit{k}-nearest neighbors
has a larger spread of absolute error.

\begin{figure}[!htb]
  \centering
  \begin{subfigure}[b]{0.49\textwidth}
    \centering
    \includegraphics[width=\textwidth]{./chapters/exp1/sfcompo_boxplots_impnull_enri.png}
    \caption{Box plots of enrichment errors using mean-imputed null values.}
    \label{fig:enriimp}
  \end{subfigure}
  \hfill
  \begin{subfigure}[b]{0.49\textwidth}
    \centering
    \includegraphics[width=\textwidth]{./chapters/exp1/sfcompo_boxplots_impnull_pcterr_enri.png}
    \caption{Box plots of enrichment percentage errors using mean-imputed null values.}
    \label{fig:enriimppct}
  \end{subfigure}
  \vskip\baselineskip
  \begin{subfigure}[b]{0.49\textwidth}
    \centering
    \includegraphics[width=\textwidth]{./chapters/exp1/sfcompo_boxplots_0null_enri.png}
    \caption{Box plots of enrichment errors using zero-imputed null values.}
    \label{fig:enri0}
  \end{subfigure}
  \hfill
  \begin{subfigure}[b]{0.49\textwidth}
    \centering
    \includegraphics[width=\textwidth]{./chapters/exp1/sfcompo_boxplots_0null_pcterr_enri.png}
    \caption{Box plots of enrichment percentage errors using zero-imputed null values.}
    \label{fig:enri0pct}
  \end{subfigure}
  \caption[Box plots of enrichment regression of \acrshort{SFCOMPO} entries]
          {Box plots of enrichment prediction errors and percentage errors for 
           each algorithm using two missing entry techniques: imputation with 
           mean values and with zero.}
  \label{fig:sfcoenri}
\end{figure}

Again, a look at the relative errors gives a different sense of these results.
Figures \ref{fig:enriimppct} and \ref{fig:enri0pct} present the percent error
statistics for the mean-imputed nulls and zero-imputed nulls test sets,
respectively.  In Figure \ref{fig:enriimppct} there are 57, 39, and 49 outliers
for the \textit{k}-nearest neighbors, decision trees, and \gls{MLL}
calculations, respectively.  In Figure \ref{fig:enri0pct} there are 44 and 23
outliers for the decision trees and \gls{MLL} calculations, respectively.

As with the burnup regression, the high percentage errors are caused by a large
overprediction of enrichments that are of low value. All of the \gls{PHWR}s fit
into this category, and removing them from the results removes the outliers
with the highest errors from the mean-imputed nulls results in Figure
\ref{fig:enriimppct}, leaving a maximum of 250\% enrichment error. This is
still a high maximum error because there are other low enrichments not
belonging to the \gls{PHWR} class that remain overpredicted.  As with the
burnup, removing \gls{PHWR}s does not significantly alter the zero-imputed
nulls results in Figure \ref{fig:enri0pct}.

Unlike the case with burnup, the decision trees algorithm gives a better
performance that is null handling-independent than the other algorithm/test set
scenarios. Although there was a slightly worse mean absolute error for the
zero-imputed nulls test set (the median absolute error was the same), the
relative error remains below 100\% for all outliers, as seen in Figure
\ref{fig:enri0pct}.  This makes the decision trees algorithm with the
zero-imputed nulls test set the best performing case for enrichment regression.
It is an unexpected result to have either one of the scikit algorithms
outperform \gls{MLL} calculations for the zero-imputed nulls test set, since
they tended to have lower errors using the mean-imputed nulls test set. But
decision trees are capable of ignoring certain features if they do not lower
the node impurity (recall Equations \ref{eq:gini} and \ref{eq:mse}), so it is
not an unexpected result without an explanation.

\begin{figure}[!htb]
  \centering
  \begin{subfigure}[b]{\textwidth}
    \centering
    \includegraphics[width=\textwidth]{./chapters/exp1/sfcompo_truey_vs_predy_impnull__enri.png}
    \caption{True versus predicted enrichment using mean-imputed null values.}
    \label{fig:yvy_enriimp}
  \end{subfigure}
  \vskip\baselineskip
  \begin{subfigure}[b]{\textwidth}
    \centering
    \includegraphics[width=\textwidth]{./chapters/exp1/sfcompo_truey_vs_predy_0null__enri.png}
    \caption{True versus predicted enrichment using zero-imputed null values.}
    \label{fig:yvy_enri0}
  \end{subfigure}
  \caption[True versus predicted enrichment of \acrshort{SFCOMPO} test cases]
          {True versus predicted enrichment for each algorithm using two missing
           entry techniques for \acrshort{SFCOMPO}: imputation with mean 
           values and with zero.}
  \label{fig:yvy_sfcoenri}
\end{figure}

To better understand the high errors in Figure \ref{fig:sfcoenri} and
especially the relative error outliers, Figure \ref{fig:yvy_sfcoenri} presents
the plots of the true \gls{U235} enrichment versus the predicted \gls{U235}
enrichment for each test sample in \gls{SFCOMPO} for each algorithm and both
imputation techniques.  The colorbar is the percentage error and was chosen
with the range up to 200\%.  This is in order to show the difference between
the large errors below the diagonal line generally having a maximum error of
100\%, whereas the errors above the diagonal line can exceed 200\%.  First,
Figure \ref{fig:yvy_enriimp} does not follow the same pattern as its burnup
counterpart (Figure \ref{fig:yvy_burnimp}) where the predictions are clustered
at a certain level due to the mean imputation.  There is still a general trend
of lower levels of enrichment being over-predicted, which are the large
relative error cases.  This is not happening because the missing values are not
strong enrichment indicators, because there is a clear effect when they are
imputed with zero, as shown in Figure \ref{fig:yvy_enri0}.  Here, the
large-error predictions are shown clustered towards the bottom for the
scikit-learn algorithms.  As with burnup (Figure \ref{fig:yvy_burn0}), this is
not the case for the \gls{MLL} calculations since the zero values are filtered. 

Overall, as with the case with burnup, the mean and median absolute errors in
Table \ref{tbl:sfcoenri} tell a much more encouraging story than the box plots
in Figure \ref{fig:sfcoenri}.  Looking at the spread of absolute errors and the
large relative errors gives the sense that the enrichment predictions are quite
poor as a whole. This does not mean it is hopeless; the decision tree approach
here warrants more investigation since the spread of errors in Figure
\ref{fig:enri0} is under $1.0\%\:{}^{235}\text{U}$. The additional details
provided by directly plotting the true versus predicted enrichment in Figure
\ref{fig:yvy_sfcoenri} is helpful in understanding how the null-value handling
methods impacted the results.


