Nuclear forensics is a nuclear security capability that is broadly defined as
material attribution in the event of a nuclear incident.  Improvement and
research is needed for technical components of this process.  One such area is
the provenance of non-detonated special nuclear material; studied here is
\gls{SNF}, which is applicable in a scenario involving the unlawful use of
commercial byproducts from nuclear power reactors.  The experimental process
involves measuring known forensics signatures to ascertain the reactor
parameters that produced the material, assisting in locating its source.  This
work proposes the use of statistical methods to determine these quantities
instead of empirical relationships. 

The purpose of this work is to probe the feasibility of this method with a
focus on field-deployable detection.  Thus, two experiments are conducted, the
first informing the second by providing a baseline of performance.  Both
experiments use simulated nuclide measurements as observations and reactor
operation parameters as the prediction goals.  First, machine learning
algorithms are employed with full-knowledge training data, i.e., nuclide
vectors from simulations that mimic lab-based mass spectrometry. The error in
the mass measurements is artificially increased to probe the prediction
performance with respect to information reduction. Second, this machine
learning workflow is performed on training data analogous to a field-deployed
gamma detector that can only measure radionuclides. The detector configuration
is varied so that the information reduction now represents decreasing detector
energy resolution.  The results are evaluated using the error of the reactor
parameter predictions.

The reactor parameters of interest are the reactor type and three quantities
that can attribute \gls{SNF}: burnup, initial \acrshort{U235} enrichment, and
time since irradiation. The algorithms used to predict these quantities are
\textit{k}-nearest neighbors, decision trees, and maximum log-likelihood
calculations. The first experiment predicts all of these quantities well using
the three algorithms, except for \textit{k}-nearest neighbors predicting time
since irradiation. For the second experiment, most of the detector
configurations predict burnup well, none of them predict enrichment well, and
the time since irradiation results perform on or near the baseline.  This
approach is an exploratory study; the results are promising and warrant further
study.

