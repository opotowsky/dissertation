Nuclear forensics is a nuclear security capability that is broadly defined as
material attribution in the event of a nuclear incident.  Improvement and
research is needed for both technical and non-technical components of this
process.  One such technical area is the provenance of non-detonated \gls{SNM};
studied here is \gls{SNF}, which is applicable in a scenario involving the
unlawful use of commercial byproducts from nuclear power reactors.  The
experimental process involves measuring known forensics signatures to ascertain
the reactor parameters that produced the material. Knowing these assists in
locating the source of the material. This work is proposing the use of
statistical methods to determine these quantities instead of empirical
relationships. 

The purpose of this work is to probe to what extent this method is feasible.
Thus, two experiments have been designed, using simulated nuclide measurements
as observations and reactor operation parameters as the prediction goals.
First, machine learning algorithms are employed with full-knowledge training
data, i.e., nuclide vectors directly from simulations.  Second, this workflow
is performed on reduced-knowledge training data, analogous to a detector that
can only measure certain radionuclides. The results are evaluated using the
performance of the reactor parameter predictions.

The reactor parameters of interest are the reactor type and three quantities of
interest describing the \gls{SNF}: burnup, initial \gls{U235} enrichment, and
time since irradiation. The algorithms used to predict these quantities are
\textit{k}-nearest neighbors, decision trees, and \gls{MLL} calculations. The
first experiment predicts all of these quantities well using the three
algorithms, except for the case where \textit{k}-nearest neighbors is
predicting time since irradiation. The second experiment has widely varying
results, but two consistent themes: the methods predict burnup very well and
enrichment poorly. This approach is an exploratory study using simple
algorithms. Even so, the results are overall promising and warrant further
study.

