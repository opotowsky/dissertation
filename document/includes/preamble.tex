% preamble.tex -- packages to include
%
% Wisconsin dissertation template
% Copyright (c) 2008 William C. Benton.  All rights reserved.
%
% This program can redistributed and/or modified under the terms
% of the LaTeX Project Public License Distributed from CTAN
% archives in directory macros/latex/base/lppl.txt; either
% version 1 of the License, or (at your option) any later version.
%
% This program includes other software that is licensed under the
% terms of the LPPL and the Perl Artistic License; see README for details.
%
% You, the user, still hold the copyright to any document you produce
% with this software (like your dissertation).


%% You should use natbib
\IfFileExists{natbib.sty}{%
\usepackage[numbers,sort]{natbib}%
}{}

%% You probably need appendix, if you want appendices
\IfFileExists{appendix.sty}{%
\usepackage{appendix}%
}{}

%% the spacing in memoir is weird, you'll need to use this
\DisemulatePackage{setspace}
%\usepackage[onehalfspacing]{setspace}
%\onehalfspacing
% The above was the standard withesis spacing. For double spacing, which is 
% required in the dissertation, do the following
\usepackage[doublespacing]{setspace}

%% List setup; the ``hanglist`` environment will allow you to have
%% nicely-typeset enumerated lists (i.e. with the numbers hanging in
%% the margins).  You need at least version 2.1 of enumitem.sty.  If
%% you don't have enumitem installed at all, hanglist will just be an
%% alias for enumerate.
\IfFileExists{enumitem.sty}{%
\usepackage[loadonly]{enumitem}[2007/06/30]%
\newlist{hanglist}{enumerate}{1}% 
\setlist[hanglist]{label=\arabic*.}%
\setlist[hanglist,1]{leftmargin=0pt}%
}{%
\newenvironment{hanglist}{\begin{enumerate}}{\end{enumerate}}%
}

%% Comment out any of these that you don't want
\usepackage{amssymb}
\usepackage{amsmath}
\usepackage{amsthm}
%\usepackage{theorem}
\usepackage{hyperref}
\usepackage{subcaption}
\usepackage{etoolbox}
\usepackage[font=itshape]{quoting}

% Other helpful things to prevent errors and warnings
\usepackage{rotating}
\apptocmd{\sloppy}{\hbadness 10000\relax}{}{}
\usepackage{float}

%% To use the glossaries acronym package, you'll need to define any acronyms you intend to 
%% use. You can define acronyms with \newacronym{label}[acronym]{written out form}
%% To refer to them in the text use \gls{label}
\usepackage[acronym,toc]{glossaries}
\makeglossaries

\IfFileExists{mathpartir.sty}{%
\usepackage{mathpartir}%
}{}

%%%%% LISTINGS package and setup
\IfFileExists{listings.sty}{%
\usepackage{listings}%
}{}

% AO adding for some tables
\newcommand\allbold[1]{{\boldmath\textbf{#1}}}

%% Custom Katy Huff Commands
%% need xspace
\usepackage{xspace}
\newcommand{\fluxJ}{\overrightarrow{J} }
\newcommand{\Cyclus}{\textsc{Cyclus}\xspace}
\newcommand{\Cyder}{\textsc{Cyder}\xspace}
\newcommand{\nucl}[2]{
\ensuremath{^{#1}}\mbox{#2}
}
\usepackage{bigints}

%% Get rid of ugly borders around PDF hyperlinks (e.g. for cross-references, bib entries, or URLs)
\hypersetup{pdfborder = 0 0 0}

%% You want microtype.
\IfFileExists{microtype.sty}{%
\usepackage[protrusion=true,expansion=true]{microtype}%
}{}

%\pagestyle{thesisdraft}

% Surround parts of graphics with box
\usepackage{boxedminipage}

%% booktabs (thx to Nate Rosenblum for bringing this beautiful package
%% to my attention)
\IfFileExists{booktabs.sty}{%
\usepackage{booktabs}%
}{}
\renewcommand{\arraystretch}{1.2}
\usepackage{multirow}
\usepackage{colortbl}

% This is now the recommended way for checking for PDFLaTeX:
\usepackage{ifpdf}

%% Avoid ugly "Type 3" fonts
\usepackage{lmodern}
\usepackage[LY1]{fontenc}

%% Substitute your favorite serif and sans fonts here....
%\IfFileExists{tgpagella.sty}{%
% TeX Gyre pagella, like Palatino
%\usepackage{tgpagella}%
%}{}

%\usepackage[LY1]{eulervm}

%\ifpdf
%\usepackage[pdftex]{graphicx}
%\else
\usepackage{graphicx}
%\fi

\usepackage{makeidx}
\makeindex

{\theoremstyle{plain}
\newtheorem{thm}{Theorem}[chapter]
\newtheorem{cor}[thm]{Corollary}
\newtheorem{define}[thm]{Definition}
\newtheorem{exmpl}[thm]{Example}
}
{\theoremstyle{remark}
\newtheorem{rmk}[thm]{Remark}
}

\newtheoremstyle{customsty1}
{3pt}%
{3pt}%
{}% --- body font
{}% --- indent amount
{\bfseries}% --- Theorem head font
{:}% --- Punctuation after head
{.5em}% --- space after head
{}% --- theorem head spec (can be left empty, meaning 'normal')

% Define 'newtheorems' that use ``customsty1''
{\theoremstyle{customsty1} 
}


%%% NB: the ``deposit'' chapter- and page- styles should conform to UW
%%% requirements.  If you are producing a pretty version of your
%%% dissertation for web use later, you will certainly want to make
%%% your own chapter and page styles.

\makechapterstyle{deposit}{%
  \renewcommand{\chapterheadstart}{}
  \renewcommand{\printchaptername}{}
  \renewcommand{\chapternamenum}{}
  \renewcommand{\printchapternum}{\parbox{2em}{\MakeLowercase{\Large\scshape\thechapter{}}} }
  \renewcommand{\afterchapternum}{}
  \renewcommand{\printchaptertitle}[1]{%
  \raggedright\Large\scshape\MakeLowercase{##1}}
  \renewcommand{\afterchaptertitle}{%
  \vskip\onelineskip \hrule\vskip\onelineskip}
}

\makepagestyle{deposit}
 
\makeatletter
 
\renewcommand{\chaptermark}[1]{\markboth{#1}{}}
\renewcommand{\sectionmark}[1]{\markboth{#1}{}}
 
\makeevenfoot{deposit}{}{}{}
\makeoddfoot{deposit}{}{}{}
\makeevenhead{deposit}{\thepage}{}{}
\makeoddhead{deposit}{}{}{\thepage}
\makeatother

%%% set up page numbering for chapter pages to satisfy UW requirements
%%% NB: You will want to delete until the ``SNIP'' mark if you are
%%% making a ``nice'' copy
\copypagestyle{chapter}{plain}
\makeoddfoot{chapter}{}{}{}
\makeevenhead{chapter}{\thepage}{}{}
\makeoddhead{chapter}{}{}{\thepage}
%%% SNIP

%%% bib nonsense
\makeatletter
\newenvironment{wb-bib}[1]{%
  \chapter*{references}
\ifnobibintoc\else 
\phantomsection 
\addcontentsline{toc}{chapter}{References} 
\fi 
\prebibhook
  \begin{bibitemlist}{#1}}{\end{bibitemlist}\postbibhook}

\AtBeginDocument{%
  \@ifpackageloaded{natbib}{% natbib is loaded
    \addtodef{\endthebibliography}{}{\vskip-\lastskip\postbibhook}
    \@ifpackagewith{natbib}{sectionbib}{% with sectionbib option
      \renewcommand{\bibsection}{\@memb@bsec}}%
      {\renewcommand{\bibsection}{\@memb@bchap}}}%
  {}
  \@ifpackagewith{chapterbib}{sectionbib}{%
    \renewcommand{\sectionbib}[2]{}
    \renewcommand{\bibsection}{\@memb@bsec}}{}
}
\makeatother

%% Katy Huff Additions
\DeclareMathOperator{\erf}{erf}
\DeclareMathOperator{\erfc}{erfc}
\usepackage{verbatim}
\usepackage{tabularx}
\setcounter{secnumdepth}{3}
\setcounter{tocdepth}{3}
\usepackage[section]{placeins}
